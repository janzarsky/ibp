%==============================================================================
% tento soubor pouzijte jako zaklad
% this file should be used as a base for the thesis
% Autoři / Authors: 2008 Michal Bidlo, 2016 Jaroslav Dytrych
% Kontakt pro dotazy a připomínky: dytrych@fit.vutbr.cz
% Contact for questions and comments: dytrych@fit.vutbr.cz
%==============================================================================
% kodovani: UTF-8 (zmena prikazem iconv, recode nebo cstocs)
% encoding: UTF-8 (you can change it by command iconv, recode or cstocs)
%------------------------------------------------------------------------------
% zpracování / processing: make, make pdf, make clean
%==============================================================================
% Soubory, které je nutné upravit: / Files which have to be edited:
%   xzarsk03-Extending-audit2allow-20-literatura-bibliography.bib - literatura / bibliography
%   xzarsk03-Extending-audit2allow-01-kapitoly-chapters.tex - obsah práce / the thesis content
%   xzarsk03-Extending-audit2allow-30-prilohy-appendices.tex - přílohy / appendices
%==============================================================================
%\documentclass[]{fitthesis} % bez zadání - pro začátek práce, aby nebyl problém s překladem
%\documentclass[english]{fitthesis} % without assignment - for the work start to avoid compilation problem
%\documentclass[zadani]{fitthesis} % odevzdani do wisu - odkazy jsou barevné
\documentclass[english,zadani]{fitthesis} % for submission to the IS FIT - links are color
%\documentclass[zadani,print]{fitthesis} % pro tisk - odkazy jsou černé
%\documentclass[zadani,cprint]{fitthesis} % pro barevný tisk - odkazy jsou černé, znak VUT barevný
%\documentclass[english,zadani,print]{fitthesis} % for the color print - links are black
%\documentclass[english,zadani,cprint]{fitthesis} % for the print - links are black, logo is color
% * Je-li práce psaná v anglickém jazyce, je zapotřebí u třídy použít 
%   parametr english následovně:
%   If thesis is written in english, it is necessary to use 
%   parameter english as follows:
%      \documentclass[english]{fitthesis}
% * Je-li práce psaná ve slovenském jazyce, je zapotřebí u třídy použít 
%   parametr slovak následovně:
%   If the work is written in the Slovak language, it is necessary 
%   to use parameter slovak as follows:
%      \documentclass[slovak]{fitthesis}
% * Je-li práce psaná v anglickém jazyce se slovenským abstraktem apod., 
%   je zapotřebí u třídy použít parametry english a enslovak následovně:
%   If the work is written in English with the Slovak abstract, etc., 
%   it is necessary to use parameters english and enslovak as follows:
%      \documentclass[english,enslovak]{fitthesis}

% Základní balíčky jsou dole v souboru šablony fitthesis.cls
% Basic packages are at the bottom of template file fitthesis.cls
% zde můžeme vložit vlastní balíčky / you can place own packages here

% Kompilace po částech (rychlejší, ale v náhledu nemusí být vše aktuální)
% Compilation piecewise (faster, but not all parts in preview will be up-to-date)
% \usepackage{subfiles}

% Nastavení cesty k obrázkům
% Setting of a path to the pictures
%\graphicspath{{obrazky-figures/}{./obrazky-figures/}}
%\graphicspath{{obrazky-figures/}{../obrazky-figures/}}

%---rm---------------
\renewcommand{\rmdefault}{lmr}%zavede Latin Modern Roman jako rm / set Latin Modern Roman as rm
%---sf---------------
\renewcommand{\sfdefault}{qhv}%zavede TeX Gyre Heros jako sf
%---tt------------
\renewcommand{\ttdefault}{lmtt}% zavede Latin Modern tt jako tt

% vypne funkci šablony, která automaticky nahrazuje uvozovky,
% aby nebyly prováděny nevhodné náhrady v popisech API apod.
% disables function of the template which replaces quotation marks
% to avoid unnecessary replacements in the API descriptions etc.
\csdoublequotesoff

% =======================================================================
% balíček "hyperref" vytváří klikací odkazy v pdf, pokud tedy použijeme pdflatex
% problém je, že balíček hyperref musí být uveden jako poslední, takže nemůže
% být v šabloně
% "hyperref" package create clickable links in pdf if you are using pdflatex.
% Problem is that this package have to be introduced as the last one so it 
% can not be placed in the template file.
\ifWis
\ifx\pdfoutput\undefined % nejedeme pod pdflatexem / we are not using pdflatex
\else
  \usepackage{color}
  \usepackage[unicode,colorlinks,hyperindex,plainpages=false,pdftex]{hyperref}
  \definecolor{links}{rgb}{0.4,0.5,0}
  \definecolor{anchors}{rgb}{1,0,0}
  \def\AnchorColor{anchors}
  \def\LinkColor{links}
  \def\pdfBorderAttrs{/Border [0 0 0] }  % bez okrajů kolem odkazů / without margins around links
  \pdfcompresslevel=9
\fi
\else % pro tisk budou odkazy, na které se dá klikat, černé / for the print clickable links will be black
\ifx\pdfoutput\undefined % nejedeme pod pdflatexem / we are not using pdflatex
\else
  \usepackage{color}
  \usepackage[unicode,colorlinks,hyperindex,plainpages=false,pdftex,urlcolor=black,linkcolor=black,citecolor=black]{hyperref}
  \definecolor{links}{rgb}{0,0,0}
  \definecolor{anchors}{rgb}{0,0,0}
  \def\AnchorColor{anchors}
  \def\LinkColor{links}
  \def\pdfBorderAttrs{/Border [0 0 0] } % bez okrajů kolem odkazů / without margins around links
  \pdfcompresslevel=9
\fi
\fi
% Řešení problému, kdy klikací odkazy na obrázky vedou za obrázek
% This solves the problems with links which leads after the picture
\usepackage[all]{hypcap}

% customization
\usepackage{tikz}

% Informace o práci/projektu / Information about the thesis
%---------------------------------------------------------------------------
\projectinfo{
    %Prace / Thesis
    %typ práce BP/SP/DP/DR  / thesis type (SP = term project)
    project={BP},
    % rok odevzdání / year of submission
    year={2018},
    % datum odevzdání / submission date
    date=\today,
    %
    %Nazev prace / thesis title
    % název práce v češtině či slovenštině (dle zadání) / thesis title in czech
    % language (according to assignment)
    title.cs={Rozšíření nástroje audit2allow pro poskytování více omezujících
    řešení},
    % název práce v angličtině / thesis title in english
    title.en={Extending audit2allow to Provide More Restrictive Solutions},
    % nastavení délky bloku s titulkem pro úpravu zalomení řádku (lze definovat
    % zde nebo níže) / setting the length of a block with a thesis title for
    % adjusting a line break (can be defined here or below)
    title.length={14.3cm},
    %
    %Autor / Author
    % jméno autora / author name
    author.name={Jan},
    % příjmení autora / author surname 
    author.surname={Žárský},
    %
    %Ustav / Department
    % doplňte příslušnou zkratku dle ústavu na zadání: UPSY/UIFS/UITS/UPGM /
    % fill in appropriate abbreviation of the department according to
    % assignment: UPSY/UIFS/UITS/UPGM
    department={UITS},
    %
    % Školitel / supervisor
    % jméno školitele / supervisor name 
    supervisor.name={Aleš},
    % příjmení školitele / supervisor surname
    supervisor.surname={Smrčka},
    %titul před jménem (nepovinné) / title before the name (optional)
    supervisor.title.p={Ing.},
    %titul za jménem (nepovinné) / title after the name (optional)
    supervisor.title.a={Ph.D.},
    %
    % Klíčová slova / keywords
    % klíčová slova v českém či slovenském jazyce / keywords in czech or slovak
    % language
    keywords.cs={SELinux, audit2allow, bezpečnost, mandatorní řízení přístupu},
    % klíčová slova v anglickém jazyce / keywords in english
    keywords.en={SELinux, audit2allow, security, mandatory access control},
    %
    % Abstrakt / Abstract
    % abstrakt v českém či slovenském jazyce / abstract in czech or slovak
    % language
    abstract.cs={Bakalářská práce rozebírá roli nástroje audit2allow při řešení
    zamítnutí přístupu systémem Security-Enhanced Linux a navrhuje rozšíření
    nástroje tak, aby poskytoval více omezující a bezpečnější řešení uživateli.
    První část popisuje základní koncepty systému SELinux. Druhá část obsahuje
    analýzu situací, kdy nástroj audit2allow poskytuje řešení, která jsou
    neefektivní a potenciálně nebezpečná. Třetí část popisuje implementaci
    vybraných rozšíření, které poskytují více omezující a bezpečnější řešení.
    Poslední část popisuje testování těchto rozšíření.},
    % abstrakt v anglickém jazyce / abstract in english
    abstract.en={The thesis analyzes the role of the audit2allow utility in
    troubleshooting Security-Enhanced Linux denials and proposes extensions that
    will provide more restrictive and more secure solutions to the user. In
    first part, basic concepts of SELinux are explained. The second part
    contains analysis of situations when audit2allow provides ineffective and
    insecure solutions. Third part describes implementation of chosen extensions
    to audit2allow that provide more restrictive and secure solutions. The last
    part describes testing of these extensions.},
    %
    % Prohlášení (u anglicky psané práce anglicky, u slovensky psané práce
    % slovensky) / Declaration (for thesis in english should be in english)
    %declaration={Prohlašuji, že jsem tuto bakalářskou práci vypracoval
    %samostatně pod vedením pana Ing. Aleše Smrčky, Ph.D.  Další informace mi
    %poskytli Miloš Malík, Petr Lautrbach, Lukáš Vrabec a Vít Mojžíš.  Uvedl
    %jsem všechny literární prameny a publikace, ze kterých jsem čerpal.},
    declaration={Hereby I declare that this bachelor's thesis was prepared as an
    original author’s work under the supervision of Ing. Aleš Smrčka, Ph.D. The
    supplementary information was provided by Miloš Malík, Petr Lautrbach, Lukáš
    Vrabec and Vít Mojžíš. All the relevant information sources, which were used
    during preparation of this thesis, are properly cited and included in the
    list of references.},
    %
    % Poděkování (nepovinné, nejlépe v jazyce práce) / Acknowledgement
    % (optional, ideally in the language of the thesis)
    %acknowledgment={V této sekci je možno uvést poděkování vedoucímu práce a
    %těm, kteří poskytli odbornou pomoc (externí zadavatel, konzultant,
    %apod.).},
    acknowledgment={Here it is possible to express thanks to the supervisor and
    to the people which provided professional help (external submitter,
    consultant, etc.).},
    %
    % Rozšířený abstrakt (cca 3 normostrany) - lze definovat zde nebo níže /
    % Extended abstract (approximately 3 standard pages) - can be defined here
    % or below
    extendedabstract={Do tohoto odstavce bude zapsán rozšířený výtah (abstrakt)
    práce v českém (slovenském) jazyce.},
}

% Rozšířený abstrakt (2-6 normostran) - lze definovat zde nebo výše / Extended
% abstract (approximately 3 standard pages) - can be defined here or above
%\extendedabstract{Do tohoto odstavce bude zapsán výtah (abstrakt) práce v
%českém (slovenském) jazyce.}

% nastavení délky bloku s titulkem pro úpravu zalomení řádku - lze definovat zde
% nebo výše / setting the length of a block with a thesis title for adjusting a
% line break - can be defined here or above
%\titlelength{14.5cm}


% řeší první/poslední řádek odstavce na předchozí/následující stránce
% solves first/last row of the paragraph on the previous/next page
\clubpenalty=10000
\widowpenalty=10000

% custom settings
\lstset{
    keepspaces=true,
    columns=flexible,
    frame=leftline,
    framesep=11pt,
}
\lstdefinelanguage{te} {
  morekeywords={
      user, role, type, range, type\_transition, allow, roles, types, alias,
      attribute, allowxperm, module, require, class, bool, if, netifcon,
      nodecon, portcon, attribute\_role, roleattribute, typeattribute, typealias
  },
  morecomment=[l]{\#},
}

\begin{document}
  % Vysazeni titulnich stran / Typesetting of the title pages
  % ----------------------------------------------
  \maketitle
  % Obsah
  % ----------------------------------------------
  \setlength{\parskip}{0pt}

  {\hypersetup{hidelinks}\tableofcontents}
  
  % Seznam obrazku a tabulek (pokud prace obsahuje velke mnozstvi obrazku, tak
  % se to hodi) List of figures and list of tables (if the thesis contains a lot
  % of pictures, it is good)
  \ifczech
    \renewcommand\listfigurename{Seznam obrázků}
  \fi
  \ifslovak
    \renewcommand\listfigurename{Zoznam obrázkov}
  \fi
  % \listoffigures
  
  \ifczech
    \renewcommand\listtablename{Seznam tabulek}
  \fi
  \ifslovak
    \renewcommand\listtablename{Zoznam tabuliek}
  \fi
  % \listoftables 

  \ifODSAZ
    \setlength{\parskip}{0.5\bigskipamount}
  \else
    \setlength{\parskip}{0pt}
  \fi

  % vynechani stranky v oboustrannem rezimu
  % Skip the page in the two-sided mode
  \iftwoside
    \cleardoublepage
  \fi

  % Text prace / Thesis text
  % ----------------------------------------------
  % Statuses:
% 0 - not started
% 1 - started, ideas collected
% 2 - less than half the length
% 3 - half the length
% 4 - full length, may need rephrasing
% 5 - done
% ==============================================================================
\chapter{Introduction}
% STATUS: 3
% motivation - longer
% how I solved it, avoid first person
% overview which chapter solves what

% establish your territory
%   state the general topic and give some background
%   provide a~review of the literature related to the topic
%   define the terms and scope of the topic
\emph{Security-Enhanced Linux} (SELinux) is a~mandatory access control mechanism
used in Linux distributions. It extends the traditional file permissions using
a~security policy that cannot be overridden by users. The \emph{audit2allow}
utility is one of several tools used by system administrators to troubleshoot
SELinux denials. Security policy developers use audit2allow to create a~basis
for security policy modules for their products. The audit2allow utility analyzes
SELinux denials and generates policy rules that can be loaded into the security
policy to allow the operations that were denied before.

% establish a~niche
%   outline the current situation
%   evaluate the current situation (advantages/ disadvantages) and identify the gap
In certain situations, the audit2allow utility fails to provide an effective and
secure solution. The utility was designed to solve problems caused by missing
rules in the security policy, but users often use audit2allow to solve problems
that should not be solved by adding new rules to the policy. As a~result, the
users end up with policy rules that give processes too much permissions, making
whole system more vulnerable.

In other situations, the audit2allow utility provides nonfunctional solutions
because it is not aware of recently added features of SELinux. There are new
policy statements that provide more granular control over given permissions. The
audit2allow utility is unable to detect that the denial was caused by these
statements and fails to provide a~working solution. Security policy developers
cannot use the audit2allow utility to generate these statements.

% introduce the current research
%   identify the importance of the proposed research
The users, not familiar with SELinux, cannot recognize the limitations of the
audit2allow utility. They either fail to solve the problem or end up with
a~workaround that is potentially insecure.
%   state the research problem/ questions
%   state the research aims and/or research objectives
This thesis aims to analyze different causes of SELinux denials and evaluate the
quality of solutions provided by the audit2allow utility. Situations, that are
best resolved using other tools, should be detected by audit2allow and the user
should be warned. Support for new SELinux features should be added to
audit2allow. As a~part of the thesis, two new features were implemented.
%   state the hypotheses

%   outline the order of information in the thesis
The second chapter of the thesis presents Security-Enhanced Linux, introduces
the SELinux policy language, describes auditing of security events, and provides
a~detailed description of the audit2allow utility. The third chapter analyzes
situations where the audit2allow utility generates nonfunctional or insecure
solutions. The fourth chapter goes through the implementation details of
selected improvements to audit2allow. The fifth one describes unit and
integration tests of implemented improvements to audit2allow.
%   outline the methodology

% ==============================================================================
\chapter{Security-Enhanced Linux and audit2allow}
% STATUS: 4
\label{selinux}
% describe SELinux, describe audit2allow, use examples, what are the
% problems of SELinux. Purpose of audit2allow, how is it used, how does it work,
% internal structure of audit2allow.

This chapter describes basic concepts of Security-Enhanced Linux, introduces the
SELinux security policy language, provides an overview of the Linux Audit
System, and describes in detail the audit2allow utility.

% ------------------------------------------------------------------------------
% TOPIC: SELinux introduction and motivation
% ----------------------------------------
% STATUS: 4
% 2.1 INTRODUCTION                                                      Y
%     2.1.1 Is SELinux useful                                           Y
% 1. Introduction
%     1.1. Benefits of running SELinux
%     1.2. Examples
\section{Security-Enhanced Linux}
Security-Enhanced Linux (SELinux) is a~mandatory access control mechanism that
consists of kernel modifications and user-space tools and is a~part of several
Linux distributions.

\subsection{Purpose of SELinux}
Without SELinux, the operating system relies on traditional access control
methods such as the file permissions. Users can grant an insecure file
permissions to others or gain access to files that they do not need
\cite{selinuxguide}:
\begin{itemize}
    \item Users can reveal sensitive information by setting world readable
        permissions on their files. For example, they can set the read
        permission for everyone on the SSH keys in the
        \texttt{\textasciitilde/.ssh/} directory.
    \item Processes can change security properties. For example, a~mail client
        can make the user's mail readable by other users.
    \item Processes inherit the user's rights. For example, each application,
        even though it may be compromised, is able to read all user's files.
\end{itemize}

SELinux enforces a~security policy that cannot be overridden by users.
Applications are allowed to perform only actions they need for normal operation,
everything else is denied by default. The applications do not need to be aware
of SELinux. When an action is denied, it is reported via an ``access denied''
error code to the application \cite{centoshowto}.

%\subsection{How Does SELinux Enforce a~Security Policy}
%
%\begin{figure}
%    \centering
%    \label{fig:policyenforcing}
%    \begin{tikzpicture}
    \usetikzlibrary{calc}
    \tikzstyle{line} = [->,>=stealth, line width=1pt]
    \tikzstyle{rec} = [rectangle, draw=black, align=center, text width=5cm]

    \node(subject) [rec] {\textbf{Subject}};
    \node(objman) [rec, below of=subject, node distance=3cm] {\textbf{Object
    Manager}\\ Queries the security server.};
    \node(secser) [rec, below of=objman, node distance=3cm]
    {\textbf{Security Server}\\ Makes decisions based on the SELinux
    policy.};

    \draw[line] (subject) -- node[midway,left] {Request access} (objman);
    \draw[line] ([xshift=-0.5cm]objman.south) -- node[midway,left] {Query
    permissions} ([xshift=-0.5cm]secser.north);
    \draw[line] ([xshift=0.5cm]secser.north) -- node[midway,right] {Get
    answer} ([xshift=0.5cm]objman.south);
\end{tikzpicture}

%    \caption{High-level process of policy enforcing}
%\end{figure}
%
%The high-level process of policy enforcing (see figure~\ref{fig:policyenforcing}):
%\begin{enumerate}
%    \item A \emph{subject} wants to perform an action upon an \emph{object}.
%    \item An \emph{Object Manager} queries the \emph{Security Server} for a
%    decision.
%    \item \emph{Security Server} consults the \emph{Security Policy}
%    and makes decision to allow or deny the action.
%\end{enumerate}
%
%For example, when process \texttt{httpd} wants to open the
%\texttt{/etc/httpd/conf/httpd.conf} file, security server in kernel allows the
%operation. It is desirable to allow \texttt{httpd} access its configuration
%files, so the SELinux policy contains rules that allow this operation. But if
%process \texttt{httpd} would want to write to the \texttt{/etc/passwd} file, the
%operation would be denied. Process \texttt{httpd} should not change the
%\texttt{/etc/passwd} file, so the rules which would allow this operation are not
%present in policy.

% ----------------------------------------
% TOPIC: SELinux components
% ----------------------------------------
% STATUS: 4
% 2.2 CORE SELINUX COMPONENTS                                           Y
%     1.3. SELinux Architecture
\subsection{SELinux Components}
SELinux is composed of kernel and user space parts \cite[pp.~19--22]{tsn}. The
main components of SELinux are shown in figure \ref{fig:selinuxcomponents}.

\begin{figure}
    \centering
    \label{fig:selinuxcomponents}
    \begin{tikzpicture}
    \usetikzlibrary{calc}
    \tikzstyle{line} = [->,>=stealth, line width=1pt]
    \tikzstyle{rec} = [rectangle, draw=black, align=center, text width=4cm,
        minimum height=2cm, minimum width=2cm]

    \draw (0,0) rectangle (10,-10);
    \draw[step=1cm,gray,very thin] (0,-20) grid (10,-10);

    \node(policysource) [rec, text width=2cm, minimum width=2.5cm]
        at (0.5,0.5) [anchor=south west]
        {\textbf{Policy Sources}};
    \node(compiling) [rec, text width=4cm, minimum width=4.5cm]
        at (0.5,-0.5) [anchor=north west]
        {\textbf{\texttt{checkmodule}\\ \texttt{semodule\_package}\\
        \texttt{semodule}}\\ Compiling and loading policy modules.};
    \node(libsepol) [rec, text width=2.5cm, minimum width=3cm, minimum
        height=6cm]
        at (9.5,-0.5) [anchor=north east]
        {\textbf{\texttt{libsepol}\\ \texttt{libsemanage}}};
    \node(policy) [rec, text width=2cm, minimum width=2.5cm]
        at (11.75,-1.5)
        {\textbf{SELinux Policy}};
    \node(semanage) [rec, below of=compiling, node distance=3cm]
        {\textbf{\texttt{semanage}}\\ Configuring parts of policy.};
    \node(policycoreutils) [rec, right of=semanage, node distance=4.5cm]
        {\textbf{\texttt{policycoreutils}}\\ Various utils such as
        \texttt{audit2allow}.};
    \node(restorecon) [rec, below of=policycoreutils, node distance=2cm]
        {\textbf{\texttt{restorecon}\\ \texttt{setfiles}}\\ File labeling
        utilities.};
    \node(libselinux) [rec, text width=8.5cm, minimum width=9cm, minimum
        height=1cm]
        at (0.5,-9.5) [anchor=south west]
        {\textbf{\texttt{libselinux}}};
    \node(selinuxfs) [rec, text width=8.5cm, minimum width=9cm, minimum
        height=1cm, below of=libselinux, node distance=2cm]
        {\textbf{SELinux Filesystem}};
    \node(kernelservices) [rec, text width=2cm, minimum width=2.5cm]
        at (0.5,-15.5) [anchor=north west]
        {\textbf{Linux Kernel Services}};
    \node(fs) [rec, text width=2.5cm, below of=kernelservices,
        node distance=3.2cm] {\textbf{Filesystems and other objects}};
    \node(lsmhooks) [rec, text width=2cm, minimum width=2.5cm]
        at (0.5,-12.5) [anchor=north west]
        {\textbf{LSM Hooks}};
    \node(avc) [rec, text width=2cm, minimum width=2.5cm, right of=lsmhooks,
        node distance=3.5cm]
        {\textbf{Access Vector Cache}};
    \node(secser) [rec, text width=2cm, minimum width=2.5cm, right of=avc,
        node distance=3.5cm]
        {\textbf{Security Server}};

    \draw[line] (policysource) -- (compiling);
    \draw[line] (compiling) -- (libsepol);
    \draw[line] (libsepol) -- (policy);
    \draw[line] (semanage) -- (libsepol);
    \draw[line] (semanage) -- (libselinux);
    \draw[line] (libselinux) -- (selinuxfs);
    \draw[line] (policycoreutils) -- (libsepol);
    \draw[line] (restorecon) -- (libsepol);
    \draw[line] (fs) -- (kernelservices);
    \draw[line] (kernelservices) -- (lsmhooks);
    \draw[line] (lsmhooks) -- (avc);
    \draw[line] (avc) -- (secser);
    \draw[line] (avc) -- (selinuxfs);
    \draw[line] (secser) -- (policy);
    \draw[line] (secser) -- (selinuxfs);

    %\draw[line] () -- ();
\end{tikzpicture}

    % TODO: add "adapted from"?
    \caption{Main SELinux components.}
\end{figure}

\begin{description}
    \item [libsepol and libsemanage] are libraries for working with the SELinux
        binary policy and the policy infrastructure. The libsepol library is
        used for example for loading policy modules into the active security
        policy. The libsemanage library is used for example for assigning
        security contexts to TCP or UDP ports.
    \item [libselinux] provides an API for implementing SELinux-aware
        applications. For example, an SELinux-aware window manager can use the
        libselinux library to compute security contexts of its objects.
    \item [checkmodule, semodule\_package, semodule] are utilities that compile
        the SELinux policy and load it into the kernel.
    \item [semanage] is an utility for configuring various parts of the policy,
        for example for setting contexts of TCP and UDP ports. It uses mainly
        the libsemanage library.
    \item [restorecon and setfiles] are utilities for restoring the default
        contexts of files (see section~\ref{filecontexts}).
    \item [policycoreutils] is a~set of various utilities for working with and
        troubleshooting of SE\-Li\-nux, for example the audit2allow utility
        described in section \ref{audit2allow}.
    \item [Modified Linux Commands] are standard Linux commands, such as ls or
        ps, modified to support SELinux.
    \item [SELinux and proc filesystem] are used by the user space tools to
        communicate with the kernel security server.
    \item [Security Server] makes security decisions. It is embedded in the
        kernel and it obtains the security policy via the user space tools.
        The security server does not enforce the decision, it only states
        whether the operation is allowed or not.
    \item [Access Vector Cache] caches security decision made by the security
        server.
    \item [Linux Security Module Hooks] call the security server.
\end{description}

% ------------------------------------------------------------------------------
\section{Basic Concepts}
% STATUS: 4

This section defines basic terms related to SELinux, such as mandatory access
control, type enforcement, multi-level and multi-category security, security
context, and others.

% ----------------------------------------
% TOPIC: Subjects and objects
% ----------------------------------------
% STATUS: 4
\subsection{Subjects and Objects}
There are two basic entities in SELinux \cite[p.~29]{tsn}:
\begin{description}
    \item [Subject] is an entity that causes information to flow among objects
        or changes the system state. Within SELinux, a~subject is an active
        process that can access objects. A~process can also be an object, for
        example when a~process is sending a~signal to another process, the
        process receiving the signal is treated as an object.
    \item [Object] is a~system resource such as a~file, a~socket, a~pipe, a~TCP
        or UDP port, a~network interface, a~semaphore or shared memory segment.
\end{description}

% ----------------------------------------
% TOPIC: MAC, DAC
% ----------------------------------------
% STATUS: 4
% 2.3 MANDATORY ACCESS CONTROL (MAC)                                    Y
\subsection{Mandatory Access Control}

SELinux provides a~mandatory access control mechanism that extends the
discretionary access control mechanisms present in the Linux kernel.

\subsubsection{Discretionary Access Control}
\emph{Discretionary access control} (DAC) is defined by \emph{Trusted Computer
System Evaluation Criteria} (TCSEC) standard \cite{orangebook}. System with DAC
must enable users to protect their data by controlling access to their data,
e.g. by setting permissions for other users or user groups. In DAC, users make
security decisions by specifying who can access their data. The disadvantage is
that users can propagate sensitive information.

Linux implements the discretionary access control. Each object has an owner
that controls the access to that object. Permissions are set in three scopes:
user, group, and others. For each scope, permissions to read, write, and execute
can be set.

\subsubsection{Mandatory Access Control}
\emph{Mandatory access control} (MAC), defined by TCSEC standard, provides more
restrictions than DAC. In this type of access control, the operating system can
prevent subjects from performing operations on objects. This is achieved by
attaching subjects and objects a~set of security attributes. When a~subject
(usually a~process) wants to perform an operation on an object (a~file,
a~directory, a~socket, etc.), the operating system first examines these
attributes. Then the security policy is used to determine whether this operation
should be allowed or not. When using MAC, users do not have the ability to
override the security policy and, for example, propagate sensitive information.

There are several implementations of MAC. The Linux kernel currently contains
several security modules implemented using the \emph{Linux Security Modules}
(LSM) framework \cite{lsmusage}. Security-Enhanced Linux, developed by National
Security Agency and Red Hat \cite{selinuxcontr}, is used in Red Hat Enterprise
Linux (RHEL), CentOS, Fedora, and Android
\cite{selinuxguide,selinuxguidefedora,selinuxandroid}. AppArmor, developed by
SUSE, is used in SUSE Linux Enterprise, openSUSE, and Ubuntu
\cite{apparmor,apparmorubuntu}. There are two other Linux security modules,
Smack and TOMOYO Linux.

\subsection{SELinux and MAC Checks}
SELinux security checks are carried out after the standard Linux DAC checks.
On an SELinux-enabled system, when a~user space process makes a~system
call, standard file permissions are checked first. Then, if the access is
allowed, the Linux Security Module hooks call security checks in SELinux.

% ----------------------------------------
% TOPIC: SELinux users
% ----------------------------------------
% STATUS: 4
% 2.4 SELINUX USERS                                                     ?
\subsection{SELinux Users}
\label{selinuxuser}
SELinux uses its own user names that are different from the standard Linux user
names \cite[p.~24]{tsn}. Each Linux user is associated to an SELinux user. For
example, Linux user \texttt{root} is mapped to an SELinux user
\texttt{unconfined\_u} on Fedora 27. There is a~special SELinux user that is
mapped to no user: \texttt{system\_u}.

\pagebreak

Available SELinux users can be listed using the \texttt{seinfo -{}-user}
command:
\begin{lstlisting}
$ seinfo --user

Users: 8
   guest_u
   root
   staff_u
   sysadm_u
   system_u
   unconfined_u
   user_u
   xguest_u
\end{lstlisting}

% ----------------------------------------
% TOPIC: RBAC
% ----------------------------------------
% STATUS: 4
% 2.5 ROLE-BASED ACCESS CONTROL (RBAC)                                  ?
\subsection{Role-Based Access Control}
\label{rbac}
SELinux uses the role-based access control (RBAC) as one of its security
mechanisms. RBAC as a~general concept is based on users, roles, permissions,
and relationships between them. RBAC defines the role-permission, user-role, and
role-role relationships.

In SELinux, each user is associated to one or more roles \cite[p.~24]{tsn}.
Each role can access only the types that are associated to that role. For
example, the \texttt{system\_u} user is associated to the
\texttt{unconfined\_r} and \texttt{system\_r} roles on Fedora 27.

Available SELinux roles can be listed using the \texttt{seinfo -{}-role}
command:
\begin{lstlisting}
$ seinfo --role

Roles: 14
   auditadm_r
   dbadm_r
   guest_r
   logadm_r
   nx_server_r
   object_r
   secadm_r
   staff_r
   sysadm_r
   system_r
   unconfined_r
   user_r
   webadm_r
   xguest_r
\end{lstlisting}

% ----------------------------------------
% TOPIC: TE
% ----------------------------------------
% STATUS: 4
% 2.6 TYPE ENFORCEMENT (TE)                                             Y
%     2.6.1 Constraints                                                 ?
%     2.6.2 Bounds                                                      N
\subsection{Type Enforcement}
\label{te}
SELinux uses type enforcement for enforcing mandatory access control
\cite[pp.~25--26]{tsn}. All subjects and objects are associated to an SELinux
type. Processes running with the same type are called a~\emph{domain}. The
SELinux policy then contains rules that allow domains access types.

\pagebreak

Available SELinux types can be listed using the \texttt{seinfo -{}-type}
command:
\begin{lstlisting}
$ seinfo --type

Types: 4845
   abrt_t
   alsa_t
   antivirus_t
   bin_t
   cluster_t
   crond_t
   ...
\end{lstlisting}

% ----------------------------------------
% TOPIC: MLS and MCS
% ----------------------------------------
% STATUS: 4
%     4.13. Multi-Level Security (MLS)
%         4.13.1. MLS and System Privileges
%         4.13.2. Enabling MLS in SELinux
%         4.13.3. Creating a~User With a~Specific MLS Range
%         4.13.4. Setting Up Polyinstantiated Directories
\subsection{Multi-Level and Multi-Category Security}
\label{mls}
In addition to the type enforcement and the role-based access control, SELinux
also supports multi-level security (MLS) and multi-category security (MCS)
\cite[pp.~48--53]{tsn}. For the purposes of MLS and MCS, the security context is
extended by a~level or range entry.

Security levels conform to the Bell-LaPadula model. For processes, the security
levels describe subjects clearance, for objects, they describe objects
classification. A~process running at a~certain security level can
\begin{itemize}
    \item read and write at their current level,
    \item read only at lower levels,
    \item write only at higher levels.
\end{itemize}
This means that processes cannot read data with a~higher security level and
cannot leak sensitive information to the lower levels. Multi-level security is
not used by default in most SELinux-enabled Linux distributions such as Fedora
or RHEL, but it is supported.

% ----------------------------------------
% TOPIC: SELinux context
% ----------------------------------------
% STATUS: 4
% 2.7 SECURITY CONTEXT                                                  Y
% 2. SELinux Contexts
\subsection{SELinux Security Context}
\label{context}
Security decisions are based on a~\emph{security context} that must be assigned
to each subject and object \cite[pp.~27--28]{tsn}. The security context is
sometimes referred to as a~\emph{security label} or just a~\emph{label}. The
security context is a~string in the following form:
\begin{lstlisting}
user:role:type[:range]
\end{lstlisting}
\begin{description}
    \item [\texttt{user}] is the SELinux user (see section \ref{selinuxuser}).
    \item [\texttt{role}] is the SELinux role used by the role-based access
        control (see section \ref{rbac}).
    \item [\texttt{type}] is the SELinux type used by the type enforcement (see
        section \ref{te}).
    \item [\texttt{range}] is used by MLS or MCS (see section \ref{mls}) and is
        optional.
\end{description}

\pagebreak

An example of subject security contexts:
\begin{lstlisting}
$ ps -eZ
LABEL                             PID TTY          TIME CMD
system_u:system_r:init_t:s0         1 ?        00:00:04 systemd
system_u:system_r:kernel_t:s0       2 ?        00:00:00 kthreadd
system_u:system_r:auditd_t:s0    1139 ?        00:00:00 auditd
system_u:system_r:alsa_t:s0      1164 ?        00:00:00 alsactl
...
\end{lstlisting}

An example of object security contexts:
\begin{lstlisting}
$ ls -Z /etc
               system_u:object_r:etc_t:s0 alsa
         system_u:object_r:cupsd_etc_t:s0 cups
          system_u:object_r:dhcp_etc_t:s0 dhcp
       system_u:object_r:passwd_file_t:s0 passwd
          system_u:object_r:net_conf_t:s0 resolv.conf
...
\end{lstlisting}

% ----------------------------------------
% TOPIC: subjects and objects, object classes, permissions, allow rule
% ----------------------------------------
% STATUS: 4
% 2.8 SUBJECTS                                                          Y
% 2.9 OBJECTS                                                           Y
%     2.9.1 Object Classes and Permissions                              Y
%     2.9.2 Allowing a~Process Access to Resources                      Y
% 4.10 OBJECT CLASS AND PERMISSION STATEMENTS                           Y
%     4.10.1 class                                                      Y
%     4.10.2 Associating Permissions to a~Class                         Y
%     4.10.3 common                                                     Y
%     4.10.4 class                                                      Y
\subsection{Object Classes}
Each object belongs to an object class. Object classes specify operations that
can be performed on the object \cite[pp.~29--30]{tsn}. For example, on Fedora
27, there are the following classes:
\begin{lstlisting}
$ seinfo --class

Classes: 97
   blk_file
   chr_file
   dbus
   dir
   fd
   file
   filesystem
   ipc
   ...
\end{lstlisting}

Each class is associated to a~set of permissions. For example, on Fedora 27,
the \texttt{node} class provides the following permissions:
\begin{lstlisting}
$ seinfo --class node -x

Classes: 1
   class node
{
	dccp_send
	enforce_dest
	tcp_recv
	rawip_send
	tcp_send
	udp_recv
	dccp_recv
	sendto
	udp_send
	recvfrom
	rawip_recv
}
\end{lstlisting}
SELinux object classes maps to the kernel object classes (files, sockets, etc.)
and user space objects (for X-Windows or D-Bus).

% ----------------------------------------
% TOPIC: labeling
% ----------------------------------------
% STATUS: 4
%     2.9.3 Labeling Objects                                            Y
%         2.9.3.1 Labeling Extended Attribute Filesystems               Y
%             2.9.3.1.1 Copying and Moving Files                        Y
%         2.9.3.2 Labeling Subjects                                     Y
%     2.9.4 Object Reuse                                                N
% 2.10 COMPUTING SECURITY CONTEXTS                                      Y
%     2.10.1 Security Context Computation for Kernel Objects            Y
%         2.10.1.1 Process                                              Y
%         2.10.1.2 Files                                                Y
%         2.10.1.3 File Descriptors                                     N
%         2.10.1.4 Filesystems                                          N
%         2.10.1.5 Network File System (nfsv4)                          N
%         2.10.1.6 INET Sockets                                         N
%         2.10.1.7 IPC                                                  N
%         2.10.1.8 Message Queues                                       N
%         2.10.1.9 Semaphores                                           N
%         2.10.1.10 Shared Memory                                       N
%         2.10.1.11 Keys                                                N
%     2.10.2 Using libselinux Functions                                 N
%         2.10.2.1 avc_compute_create and security_compute_create       N
%         2.10.2.2 avc_compute_member and security_compute_member       N
%         2.10.2.3 security_compute_relabel                             N
%     2.2. SELinux Contexts for Processes
%     2.3. SELinux Contexts for Users
\subsection{Labeling Subjects and Objects}
Security contexts for subjects and objects are computed by the kernel security
server using several policy statements \cite[pp.~31--33]{tsn}.

\subsubsection{Labeling Processes}
The first init process usually transitions to its own unique domain, for example
\texttt{init\_t}. On fork, the child process inherits the security context of
its parent. On exec, the child process may transition to a~different security
context. This is achieved by type transition policy statements. SELinux-aware
processes may change their context by calling the \texttt{setcon()} or
\texttt{setexeccon()} functions from the libselinux library.

\subsubsection{Labeling Files}
Security context for files is computed as follows:
\begin{description}
    \item [\texttt{user}] is inherited from the creating process.
    \item [\texttt{role}] defaults to \texttt{object\_r} unless modified by
        a~\texttt{role\_transition} statement.
    \item [\texttt{type}] defaults to the type of the parent directory unless
        modified by a~\texttt{type\_transition} statement.
\end{description}
The file contexts are covered in depth in section \ref{filecontexts}.

% ----------------------------------------
% TOPIC: type transitions
% ----------------------------------------
% STATUS: 4
% 2.12 DOMAIN AND OBJECT TRANSITIONS                                    Y
%     2.12.1 Domain Transition                                          Y
%         2.12.1.1 Type Enforcement Rules                               Y
%     2.12.2 Object Transition                                          Y
%     2.1. Domain Transitions
%     4.14. File Name Transition
\subsection{Type Transitions}
\label{typetransitions}
To run different processes in different domains, there needs to be a~way how to
\emph{transition} a~process from one domain to another. To attach a~specific
label to an object, the object needs to be transitioned from one type to
another. Both transitions can be achieved using the \texttt{type\_transition}
statement.

\subsubsection{Domain Transition}
Starting a~new process with a~different security context is called domain
transition \cite[pp.~43--47]{tsn}. For example, the \texttt{systemd} process
running as \texttt{init\_t} needs to start the Apache HTTP Server as
\texttt{httpd\_t}. Apache executables are labeled \texttt{httpd\_exec\_t}. The
following policy rule allows the transition:
\begin{lstlisting}[language=te]
type_transition init_t httpd_exec_t:process httpd_t;
\end{lstlisting}
The \texttt{systemd} process does not need to be aware of SELinux. The
\texttt{type\_transition} rule in the policy will cause the \texttt{exec} call
to automatically perform the transition. There are conditions that need to be
met before a~domain transition can happen:
\begin{enumerate}
    \item The source domain has a~permission to transition into the target
        domain. For example:
\begin{lstlisting}[language=te]
allow init_t httpd_t:process transition;
\end{lstlisting}
    \item The source domain has a~permission to read and execute the binary. For
        example:
\begin{lstlisting}[language=te]
allow init_t httpd_exec_t:file { execute read getattr };
\end{lstlisting}
    \item The context of the executable needs to be set as an entry point into
        the target domain. For example:
\begin{lstlisting}[language=te]
allow httpd_t httpd_exec_t:file entrypoint;
\end{lstlisting}
\end{enumerate}

\subsubsection{Object Transition}
When a~new object is created, it inherits the security context of its parent. If
a~different context of the object is required, the object transition must be
used \cite[pp.~47--48]{tsn}. For example when an X server creates a~file in the
\texttt{/tmp} directory (which has the \texttt{tmp\_t} context), it gets the 
\texttt{user\_tmp\_t} context. This is achieved by the following
\texttt{type\_transition} rule:
\begin{lstlisting}[language=te]
type_transition xserver_t tmp_t:file user_tmp_t;
\end{lstlisting}
The X server does not need to be aware of SELinux, the kernel computes the label
automatically.

% ----------------------------------------
% TOPIC: SELinux modes
% ----------------------------------------
% STATUS: 4
% 2.15 SELINUX PERMISSIVE AND ENFORCING MODES                           Y
%     1.4. SELinux States and Modes
\subsection{SELinux Modes of Operation}

SELinux has three modes of operation \cite{selinuxguide}. The default mode is
\emph{enforcing}. In this mode, everything, which is not allowed by the policy,
is denied. When a~process tries to perform an action, which is not allowed by
the policy, it is logged. In \emph{permissive} mode, SELinux is not enforcing
the policy, it only logs actions. In \emph{disabled} mode, SELinux is turned
off.

Running SELinux in permissive mode is useful for catching AVC denials that can
be analyzed using the audit2allow utility. To perform a~single task (for example
to save a~file), several SELinux checks are usually needed. When running in the
enforcing mode, only the first denial would be logged and the troubleshooting
would be more difficult.

% ------------------------------------------------------------------------------
% TOPIC: Policy language
% ----------------------------------------
% STATUS: 4
% 4.1 INTRODUCTION                                                      Y
%     4.1.1 CIL Overview                                                Y
% 4.2 KERNEL POLICY LANGUAGE                                            Y
%     4.2.1 Policy Source Files                                         Y
%     4.2.2 Conditional, Optional and Require Statement Rules           Y
%     4.2.3 MLS Statements and Optional MLS Components                  N
%     4.2.4 General Statement Information                               Y
%     4.2.5 Section Contents                                            N
% 2.14 TYPES OF SELINUX POLICY                                          Y
%     2.14.1 Example Policy                                             N
%     2.14.2 Reference Policy                                           Y
%     2.14.3 Policy Functionality Based on Name or Type                 Y
%     2.14.4 Custom Policy                                              N
%     2.14.5 Monolithic Policy                                          Y
%     2.14.6 Loadable Module Policy                                     Y
%         2.14.6.1 Optional Policy                                      N
%     2.14.7 Conditional Policy                                         ?
%     2.14.8 Binary Policy                                              Y
%     2.14.9 Policy Versions                                            Y
\section{SELinux Policy}
\label{policy}
Security decisions made by the security server in the kernel are resolved using
the SELinux policy. This section describes the most important SELinux policy
statements. The primary output of the audit2allow utility are policy statements.

SELinux supports either monolithic (compiled from a~single source file) or
modular policy. Modular policy, which is used in Fedora and RHEL, consists of
a~mandatory base policy source file and loadable modules. In Fedora, almost
each module contains a~policy for one application or service, such as the
\texttt{apache} or \texttt{xserver} module. The audit2allow utility can be used
to create a~loadable policy module.

SELinux policy statements start with a~statement keyword, usually followed by
several identifiers and a~semicolon at the end. Comments start with a~``\#''.
An example of an allow rule:
\begin{lstlisting}[language=te]
# This is an allow rule
allow httpd_t httpd_exec_t: file { ioctl read getattr lock execute open };
\end{lstlisting}

% ----------------------------------------
% TOPIC: user, role and type statements
% ----------------------------------------
% STATUS: 4
% 4.3 POLICY CONFIGURATION STATEMENTS                                   ?
%     4.3.1 policycap                                                   ?
% 4.4 DEFAULT OBJECT RULES                                              Y
%     4.4.1 default_user                                                ?
%     4.4.2 default_role                                                ?
%     4.4.3 default_type                                                ?
%     4.4.4 default_range                                               ?
% 4.5 USER STATEMENTS                                                   ?
%     4.5.1 user                                                        ? 
% 4.6 ROLE STATEMENTS                                                   ?
%     4.6.1 role                                                        ?
%     4.6.2 attribute_role                                              ?
%     4.6.3 roleattribute                                               ?
%     4.6.4 allow                                                       ?
%     4.6.5 role_transition                                             ?
%     4.6.6 dominance                                                   ?
% 4.7 TYPE STATEMENTS                                                   Y
%     4.7.1 type                                                        Y
%     4.7.2 attribute                                                   Y
%     4.7.3 typeattribute                                               Y
%     4.7.4 typealias                                                   Y
%     4.7.5 permissive                                                  Y
%     4.7.6 type_transition                                             Y
%     4.7.7 type_change                                                 N
%     4.7.8 type_member                                                 N
% user role attribute_role roleattribute allow role_transition type
% attribute typeattribute
% user -- role -- type
% role - group: attribute_role, rule: roleattribute
% type - group: attribute, typeattribute

\subsection{User, Role and Type Statements}
\label{userroletype}
To support mechanisms such as the type enforcement, the role-based access
control, and multi-level and multi-category security, SELinux assigns security
contexts to subjects and objects. A~security context is a~combination of a~user,
a~role, a~type, and optionally range identifiers (see section \ref{context}).
This section describes policy statements that declare these identifiers.

SELinux users are declared using the \texttt{user} statement. The users are
assigned one or more roles. SELinux roles are declared using the \texttt{role}
statement. The roles are assigned types that they can access. SELinux types are
declared using the \texttt{type} statement.

\begin{figure}
    \centering
    \label{fig:userroletype}
    \begin{tikzpicture}
    \usetikzlibrary{calc}
    \tikzstyle{arrow} = [->,>=stealth, line width=1pt]
    \tikzstyle{darrow} = [<->,>=stealth, line width=1pt]
    \tikzstyle{rec} = [rectangle, draw=black, align=center, text width=3.5cm,
        minimum height=1cm, minimum width=4cm, fill=white]

    \node(user) [rec, text width=1.5cm, minimum width=2cm]
        at (0,-0.5) [anchor=north west]
        {\textbf{\texttt{user}}};

    \node(role) [rec, text width=1.5cm, minimum width=2cm]
        at (5,-0.5) [anchor=north west]
        {\textbf{\texttt{role}}};

    \node(type) [rec, text width=1.5cm, minimum width=2cm]
        at (10,-0.5) [anchor=north west]
        {\textbf{\texttt{type}}};

    \node(roleattr) [rec, text width=3cm, minimum width=3.5cm]
        at (4.25,-2.5) [anchor=north west]
        {\textbf{\texttt{attribute\_role}}};

    \node(typeattr) [rec, text width=2cm, minimum width=2.5cm]
        at (9.75,-2.5) [anchor=north west]
        {\textbf{\texttt{attribute}}};

    \node(typealias) [rec, text width=2cm, minimum width=2.5cm]
        at (9.75,1.5) [anchor=north west]
        {\textbf{\texttt{typealias}}};

    \draw[arrow] (user) -- node[pos=0.5,above] {is associated to} (role);
    \draw[arrow] (role) -- node[pos=0.5,above] {is associated to} (type);
    \draw[darrow] (typealias) -- (type);
    \draw[arrow] (role) -- node[pos=0.5,left] {\texttt{roleattribute}}
        (roleattr);
    \draw[arrow] (type) -- node[pos=0.5,left] {\texttt{typeattribute}}
        (typeattr);
\end{tikzpicture}

    \caption{Relationship of the user, role, and type statements.}
\end{figure}

The roles can be grouped together using the \texttt{attribute\_role} and
\texttt{roleattribute} statements. The types can be grouped together using the
\texttt{attribute} and \texttt{typeattribute} statements. Type aliases can be
defined using the \texttt{typealias} statements. The relationship between
various statements is shown in figure \ref{fig:userroletype}.

\subsubsection{User Statements}
The \texttt{user} statement declares an identifier for an SELinux user. Syntax:
\begin{lstlisting}[language=te]
user seuser_id roles role_id;
\end{lstlisting}
\begin{description}
    \item [\texttt{seuser\_id}] is an SELinux user identifier.
    \item [\texttt{role\_id}] is one or more role identifiers.
\end{description}
An example from Fedora 27:
\begin{lstlisting}[language=te]
user staff_u roles { system_r unconfined_r sysadm_r staff_r };
\end{lstlisting}

\subsubsection{Role Statements}
The \texttt{role} statement declares an identifier for an SELinux role and
optionally associates the role to one or more types. Syntax:
\begin{lstlisting}[language=te]
role role_id;
role role_id types type_id;
\end{lstlisting}
\begin{description}
    \item [\texttt{role\_id}] is an SELinux role identifier.
    \item [\texttt{type\_id}] is one or more type identifiers.
\end{description}
An example from Fedora 27:
\begin{lstlisting}[language=te]
role auditadm_r types { auditadm_t auditadm_screen_t auditadm_su_t
    auditadm_sudo_t chkpwd_t updpwd_t exim_t auditctl_t auditd_t
    mailman_mail_t user_mail_t postfix_postdrop_t postfix_postqueue_t
    qmail_inject_t qmail_queue_t run_init_t user_tmp_t vlock_t };
\end{lstlisting}

\subsubsection{Type Statements}
The \texttt{type} statement declares an identifier for an SELinux type. Type
identifiers usually ends with '\texttt{\_t}' to distinguish them from attribute
identifiers. Syntax:
\begin{lstlisting}[language=te]
type type_id;
type type_id, attribute_id;
type type_id alias alias_id;
type type_id alias alias_id, attribute_id;
\end{lstlisting}
\begin{description}
    \item [\texttt{type\_id}] is an SELinux type identifier.
    \item [\texttt{alias\_id}] is one or more optional aliases declared by the
        \texttt{typealias} statement. Multiple aliases must be enclosed in
        braces.
    \item [\texttt{attribute\_id}] is one or more optional attributes declared
        by the \texttt{attribute} statement. Multiple attributes must be
        separated by a~comma.
\end{description}
An example from Fedora 27:
\begin{lstlisting}[language=te]
type httpd_sys_content_t alias { httpd_fastcgi_content_t
    httpd_httpd_sys_script_ro_t httpd_fastcgi_script_ro_t },
    httpdcontent, httpd_content_type, entry_type, exec_type, file_type,
    non_auth_file_type, non_security_file_type;
\end{lstlisting}

\subsubsection{Other Statements}
The \texttt{attribute\_role} statement declares an identifier for a~group of
role identifiers. Syntax:
\begin{lstlisting}[language=te]
attribute_role attribute_id;
\end{lstlisting}
The \texttt{roleattribute} statement associates roles to role attributes.
Syntax:
\begin{lstlisting}[language=te]
roleattribute role_id attribute_id;
\end{lstlisting}
The \texttt{attribute} statement declares an identifier for a~group of type
identifiers. Syntax:
\begin{lstlisting}[language=te]
attribute attribute_id;
\end{lstlisting}
The \texttt{typeattribute} statement associates types to attributes. Syntax:
\begin{lstlisting}[language=te]
typeattribute type_id attribute_id;
\end{lstlisting}
The \texttt{typealias} statement declares type aliases. Syntax:
\begin{lstlisting}[language=te]
typealias type_id alias alias_id;
\end{lstlisting}

% ----------------------------------------
% TOPIC: AV rules
% ----------------------------------------
% STATUS: 4
% 4.9 ACCESS VECTOR RULES                                               Y
%     4.9.1 allow                                                       Y
%     4.9.2 dontaudit                                                   Y
%     4.9.3 auditallow                                                  Y
%     4.9.4 neverallow                                                  Y
\subsection{Access Vector Rules}
\label{avrules}
The access vector rules support the type enforcement within SELinux. They
control which access processes get. The audit2allow utility generates access
vector rules as an output.

The \texttt{allow} rule grants an access to an object. Syntax:
\begin{lstlisting}[language=te]
allow source_type target_type:obj_class perm_set;
\end{lstlisting}
\begin{description}
    \item [\texttt{source\_type}] represents one or more type or attribute
        identifiers (see section \ref{userroletype}). This field identifies the
        subject that is performing the operation.
    \item [\texttt{target\_type}] represents one or more type or attribute
        identifiers.  This field identifies the object that is being accessed.
        When the target type is the same as the source type, the \texttt{self}
        keyword can be used instead of the target type.
    \item [\texttt{obj\_class}] represents one or more object classes (for
        example \texttt{file} or \texttt{tcp\_socket}).
    \item [\texttt{perm\_set}] represents one or more permissions (for example
        \texttt{read} or \texttt{connectto}).
\end{description}
An example:
\begin{lstlisting}[language=te]
allow httpd_t samba_share_t:file { getattr open read };
\end{lstlisting}
In this example, processes running as \texttt{httpd\_t} are allowed to
\texttt{getattr}, \texttt{open}, and \texttt{read} files labeled as
\texttt{samba\_share\_t}.

There are three other AV rules that follow the syntax pattern of the
\texttt{allow} rule:
\begin{description}
    \item [\texttt{dontaudit}] stops auditing (logging) of the denials. It is
        used when the denial is expected to happen and does not cause any
        issues. The \texttt{dontaudit} rules help to keep the audit logs clean.
    \item [\texttt{auditallow}] audits the event. The \texttt{auditallow} rule
        itself does not allow the operation, so the rule must appear together
        with a~standard \texttt{allow} rule.
    \item [\texttt{neverallow}] is a~compiler statement that stops the
        compilation of the policy if an \texttt{allow} rule with same arguments
        is found somewhere in the policy.  It is used for marking rules that may
        be insecure.
\end{description}
Internally, access vectors defined by AV rules are stored in the memory as bit
arrays that are 32~bits long. Because of this limitation, object classes cannot
have more than 32 different permissions. Extended permission AV rules were
introduced to overcome this issue.

% ----------------------------------------
% TOPIC: Extended permission AV rules
% ----------------------------------------
% STATUS: 4
\subsection{Extended Permission Access Vector Rules}
\label{extavrules}

Since policy version 30, there are extended permission access vector rules that
expand the permission sets. Standard access vector rules operate with 32~bit
permission sets, the extended permission AV rules add arbitrary number of
256~bit increments. The extended permission AV rules are currently (as of policy
version 31) used only for filtering of ioctl system calls, but they provide
generic tool that can be used for more granular control over an operation in the
future \cite{selinuxmailxperms}.

Support for the extended permission AV rules by the audit2allow utility was
implemented as a~part of this thesis. Syntax of extended permission AV rules
\cite{xpermrules}:
\begin{lstlisting}[language=te]
rule_name source_type target_type : obj_class operation xperm_set;
\end{lstlisting}
\begin{description}
    \item [\texttt{rule\_name}] is one of the following: \texttt{allowxperm},
        \texttt{dontauditxperm}, \texttt{auditallowxperm}, or
        \texttt{neverallowxperm}. The meaning is the same as with standard AV
        rules. The \texttt{allow\-xperm} rule allows the access, the
        \texttt{dontauditxperm} rule denies and logs the access, the
        \texttt{auditallowxperm} rule logs the access, and the
        \texttt{neverallowxperm} rules is a~compiler statement to prevent
        insecure rules from appearing in the policy.
    \item [\texttt{source\_type}, \texttt{target\_type}, \texttt{obj\_class}]
        are a~source type, a~target type, and an object class, the same as with
        a~standard AV rule.
    \item [\texttt{operation}] is a~single keyword defining the operation to be
        implemented by the rule. As of policy version 31, only the
        \texttt{ioctl} operation is supported. In contrast to permissions in
        standard access vector rules, each extended permission AV rule has only
        one operation (standard AV rules can have a~lot of permissions).
    \item [\texttt{xperm\_set}] are extended permissions represented by numeric
        values. The meaning of the values depends on the operation. The values
        can be written in a~decimal or hexadecimal form, for example \texttt{42}
        or \texttt{0x2a}. Multiple values must be separated by a~space and
        enclosed in braces, for example \texttt{\{ 1 2 3 \}}. Value ranges are
        supported, for example \texttt{50-60} (both 50 and 60 are included in
        the range).  To allow all values except for those explicitly listed, the
        complement operator can be used, for example
        \texttt{\textasciitilde \{ 1 2 3 \}}.
\end{description}

An example of an extended permission AV rule:
\begin{lstlisting}[language=te]
allowxperm my_app_t my_socket_t : tcp_socket ioctl { 20 30 0x40 50-60 };
\end{lstlisting}
This rule allows a~process running as \texttt{my\_app\_t} to call \texttt{ioctl}
on a~TCP socket labeled \texttt{my\_socket\_t} with parameters 20, 30, 64, or
any number from 50 to 60.

\subsubsection{Filtering the ioctl System Call}
Filtering ioctl calls is as of policy version 31 the only implementation of
extended permission AV rules. The ioctl system call accepts three parameters:
a~file descriptor, a~request number, and a~pointer \cite{ioctl}. The extended
permission AV rules allow filtering based on the request number. For ioctl
calls, the \texttt{operation} keyword is \texttt{ioctl} and numbers in the
\texttt{xperm\_set} represent request numbers.

When there is only an \texttt{allow} rule for a~particular source and a~target
context and object class, all ioctl calls are allowed. With an additional
\texttt{allowxperm} rule, only the ioctl calls with parameters allowed by the
\texttt{allowxperm} rules are allowed. The \texttt{allowxperm} rule alone has no
effect, for ioctl filtering, both \texttt{allow} and \texttt{allowxperm} rules
must be present.

% ----------------------------------------
% TOPIC: Policy modules
% ----------------------------------------
% STATUS: 4
% 4.17 MODULAR POLICY SUPPORT STATEMENTS                                Y
%     4.17.1 module                                                     Y
%     4.17.2 require                                                    Y
%     4.17.3 optional                                                   Y
\subsection{Policy Modules}
\label{modules}

The \texttt{module} and \texttt{require} statements are used to support policy
modules. The audit2allow utility is able to generate these statements when
specified by command-line options. Each policy module must start with the
\texttt{module} statement. Syntax:
\begin{lstlisting}[language=te]
module module_name version;
\end{lstlisting}
\begin{description}
    \item [\texttt{module\_name}] is the name of the module.
    \item [\texttt{version}] is the version number in format \texttt{X.Y.Z}.
\end{description}
This name is used to refer to the module when using user space utilities. For
example this command is used to remove a~module from the policy:
\begin{lstlisting}
$ semodule -r module_name
\end{lstlisting}

The \texttt{require} statement indicates which parts of the policy are imported
from other modules or the base policy. Syntax:
\begin{lstlisting}[language=te]
require { require_list }
\end{lstlisting}
\begin{description}
    \item [\texttt{require\_list}] represents one or more keywords followed by
        an identifier separated by a~semicolon. The valid keywords are:
        \texttt{role}, \texttt{type}, \texttt{attribute}, \texttt{user},
        \texttt{bool}, \texttt{category}, \texttt{sensitivity}, \texttt{class}.
\end{description}

An example of the \texttt{module} and \texttt{require} statements:
\begin{lstlisting}[language=te]
module my_module 1.2.0;

require {
    type nscd_t, nscd_var_run_t;
    class nscd { getserv getpwd getgrp gethost shmempwd shmemgrp
        shmemhost shmemserv };
}
\end{lstlisting}
When loading this module, the \texttt{nscd\_t} and \texttt{nscd\_var\_run\_t}
types and the \texttt{nscd} class with specified permissions must be defined
somewhere in the policy (either in the base policy or in another policy module).

% ----------------------------------------
% TOPIC: Conditional policy
% ----------------------------------------
% STATUS: 4
% 4.11 CONDITIONAL POLICY STATEMENTS                                    Y
%     4.11.1 bool                                                       Y
%     4.11.2 if                                                         Y
%     4.6. Booleans
%         4.6.1. Listing Booleans
%         4.6.2. Configuring Booleans
%         4.6.3. Shell Auto-Completion
\subsection{Conditional Policy}
\label{booleans}
SELinux policy allows turning on and off a~set of policy statements without the
need for reloading policy. Conditional policy is defined using the \texttt{bool}
statement that defines a~condition. Then an~\texttt{if}/\texttt{else} construct
is used to mark statements that depend on the condition. An example:
\begin{lstlisting}[language=te]
bool allow_execmem false;

if (allow_execmem) {
    allow sysadm_t self:process execmem;
}
\end{lstlisting}
Booleans can be turned on and off using the \texttt{semanage boolean} command.
The audit2allow utility is able to detect that certain AVC denials can be solved
by turning on a~boolean.

% ----------------------------------------
% TOPIC: Networking
% ----------------------------------------
% STATUS: 4
% 2.21 SELINUX NETWORKING SUPPORT                                       N
%     2.21.1 SECMARK                                                    N
%     2.21.2 NetLabel - Fallback Peer Labeling                          N
%     2.21.3 NetLabel - CIPSO                                           N
%     2.21.4 Labeled IPSec                                              N
%     2.21.4.1 Configuration Examples                                   N
% 4.16 NETWORK LABELING STATEMENTS                                      Y
%     4.16.1 IP Address Formats                                         Y
%         4.16.1.1 IPv4 Address Format                                  Y
%         4.16.1.2 IPv6 Address Formats                                 Y
%     4.16.2 netifcon                                                   Y
%     4.16.3 nodecon                                                    Y
%     4.16.4 portcon                                                    Y
\subsection{Labeling Network Objects}

The SELinux policy supports labeling of the following network objects:
\begin{itemize}
    \item TCP and UDP ports identified by a~number,
    \item network nodes represented by IP addresses and subnet masks,
    \item network interfaces (e.g. \texttt{eth0}).
\end{itemize}
The audit2allow utility can be extended to suggest changing labels of network
objects, see section \ref{networkobjects}.

\subsubsection{Network Interfaces}
The \texttt{netifcon} statement labels network interface statements. Syntax:
\begin{lstlisting}[language=te]
netifcon netif_id netif_context packet_context
\end{lstlisting}
\begin{description}
    \item [\texttt{netif\_id}] is the name of the network interface (e.g.
        \texttt{eth0}).
    \item [\texttt{netif\_context}] is the security context of the interface.
    \item [\texttt{packet\_context}] is the security context of the packets.
\end{description}
An example:
\begin{lstlisting}[language=te]
netifcon eth0 system_u:object_r:netif_t:s0 system_u:object_r:netif_t:s0
\end{lstlisting}

\subsubsection{Network Nodes}
The \texttt{nodecon} statement labels network addresses. Syntax:
\begin{lstlisting}[language=te]
nodecon subnet netmask node_context
\end{lstlisting}
\begin{description}
    \item [\texttt{subnet}] is the IP address of the subnet.
    \item [\texttt{netmask}] is the subnet mask.
    \item [\texttt{node\_context}] is the security context of the node.
\end{description}
An example:
\begin{lstlisting}[language=te]
nodecon ff00:: ff00:: system_u:object_r:multicast_node_t:s0
\end{lstlisting}

\subsubsection{Network Ports}
The \texttt{portcon} statement labels TCP and UDP ports. Syntax:
\begin{lstlisting}[language=te]
portcon protocol port_number port_context
\end{lstlisting}
\begin{description}
    \item [\texttt{protocol}] is either \texttt{udp} or \texttt{tcp}.
    \item [\texttt{port\_number}] is a~port number or a~range of port numbers.
    \item [\texttt{port\_context}] is the security context of the port.
\end{description}
An example:
\begin{lstlisting}[language=te]
portcon tcp 22 system_u:object_r:ssh_port_t:s0
\end{lstlisting}

% ----------------------------------------
% TOPIC: Interfaces
% ----------------------------------------
%\subsection{Interfaces}
%\label{interfaces}

% ----------------------------------------
% TOPIC: File contexts
% ----------------------------------------
% STATUS: 4
%     4.7. SELinux Contexts – Labeling Files
%         4.7.1. Temporary Changes: chcon
%         4.7.2. Persistent Changes: semanage fcontext
%     4.8. The file_t and default_t Types
%     4.9. Mounting File Systems
%         4.9.1. Context Mounts
%         4.9.2. Changing the Default Context
%         4.9.3. Mounting an NFS Volume
%         4.9.4. Multiple NFS Mounts
%         4.9.5. Making Context Mounts Persistent
%     4.10. Maintaining SELinux Labels
%         4.10.1. Copying Files and Directories
%         4.10.2. Moving Files and Directories
%         4.10.3. Checking the Default SELinux Context
%         4.10.4. Archiving Files with tar
%         4.10.5. Archiving Files with star
% TODO: chapters from selinux configuration section
\section{File Contexts}
\label{filecontexts}
When accessing files, SELinux relies on the labels stored with those files to
make a~security decision. SELinux labels can be viewed using the \texttt{ls -Z}
command:
\begin{lstlisting}
$ ls -Z
unconfined_u:object_r:user_home_t:s0    testdir
unconfined_u:object_r:user_home_t:s0    testfile
\end{lstlisting}
The labels are stored in \emph{extended attributes} in the security namespace
\cite{xattrman}. The extended attributes associated with a~file can be viewed
using the getfattr command:
\begin{lstlisting}
$ getfattr -n security.selinux testfile
# file: testfile
security.selinux="unconfined_u:object_r:user_home_t:s0"
\end{lstlisting}
Mislabeled files often causes AVC denials that should not be solved using the
audit2allow utility and audit2allow should detect these situations. Proper
solutions should be presented to the user. See section \ref{mislabeled}.

\subsection{Temporary Changes}
The chcon command changes the security context of files \cite{selinuxguide}. The
user must have a~permission to relabel files. The changes made by chcon are
overwritten by a~file system relabel or running of restorecon.

\subsection{Type Transition}
To specify the context of files created by processes, the
\texttt{type\_transition} rules are used. For example, when a~process running
with the \texttt{httpd\_t} context creates a~file in a~directory with the
\texttt{var\_run\_t} context, the file will get the
\texttt{httpd\_var\_run\_t} context:
\begin{lstlisting}[language=te]
type_transition httpd_t var_run_t:file httpd_var_run_t;
\end{lstlisting}
Type transitions are explained in section \ref{typetransitions}.

\subsection{File Context Configuration Files}
There are situations when files get a~label that is different than the default
one \cite{selinuxguide}:
\begin{enumerate}
    \item When moving files, the label is preserved. This does not happen when
        copying files because a~new file is always created.
    \item When SELinux is disabled, labels are not assigned to files.
    \item When the policy is changed (for example when a~module is unloaded),
        there may be some files left with a~type that is no longer defined in
        the policy.
\end{enumerate}
For these situations, there is the~\texttt{file\_contexts} file which specifies
default contexts for each file based on its path. For example:
\begin{lstlisting}
/run/.*         --  system_u:object_r:var_run_t:s0
/var/.*	        --  system_u:object_r:var_t:s0
/etc/.*	        --  system_u:object_r:etc_t:s0
/lib/.*	        --  system_u:object_r:lib_t:s0
/usr/.*\.cgi    --  system_u:object_r:httpd_sys_script_exec_t:s0
/root(/.*)?     --  system_u:object_r:admin_home_t:s0
/dev/[0-9].*    -c  system_u:object_r:usb_device_t:s0
/dev/.*tty[^/]* -c  system_u:object_r:tty_device_t:s0
\end{lstlisting}
The \texttt{-{}-} means that the context should be applied to all file types
(e.g., files, directories, sockets). The \texttt{-c} means that the context
should be applied only when the file is a~character device. Utilities such as
restorecon and setfiles uses the \texttt{file\_contexts} configuration file to
relabel files on the filesystem.

\subsection{Building File Context Configuration Files}
Utilities such as restorecon and setfiles use several files to restore default
contexts of the files \cite[pp.~165--167]{tsn}:
\begin{itemize}
    \item The \texttt{file\_contexts} file contains default contexts for files.
    \item The \texttt{file\_contexts.homedirs} file contains default contexts
        for files inside the user home directories.
    \item The \texttt{file\_contexts.local} file contains local modifications of
        the default file contexts.
    \item The \texttt{file\_contexts.subs} and
        \texttt{file\_contexts.subs\_dist} files contain file name
        substitutions. For example, these files can specify that
        \texttt{/usr/lib64} should be treated the same way as \texttt{/usr/lib}.
\end{itemize}

\begin{figure}
    \centering
    \label{fig:filecontexts}
    \begin{tikzpicture}
    \usetikzlibrary{calc}
    \tikzstyle{arrow} = [->,>=stealth, line width=1pt]
    \tikzstyle{darrow} = [<->,>=stealth, line width=1pt]
    \tikzstyle{rec} = [rectangle, draw=black, align=center, text width=3.5cm,
        minimum height=1cm, minimum width=4cm, fill=white]

    \draw (0,-6.5) rectangle (11,-9.5);
    %\draw[step=1cm,gray,very thin] (0,0) grid (11,-8);

    \node(fcfiles) [rec, text width=2cm, minimum width=2.5cm]
        at (3.75,-0.5) [anchor=north west]
        {\textbf{\texttt{.fc} files}};

    \node(fctemp) [rec, text width=5cm, minimum width=5.5cm]
        at (2.25,-2.5) [anchor=north west]
        {\textbf{\texttt{file\_contexts.template}}};

    \node(fc) [rec, text width=3cm, minimum width=3.5cm]
        at (0.5,-7) [anchor=north west]
        {\textbf{\texttt{file\_contexts}}};

    \node(hdtemp) [rec, text width=3.5cm, minimum width=4cm]
        at (5.75,-4.5) [anchor=north west]
        {\textbf{\texttt{homedir\_template}}};

    \node(fchd) [rec, text width=5cm, minimum width=5.5cm]
        at (5,-7) [anchor=north west]
        {\textbf{\texttt{file\_contexts.homedirs}}};

    \node(text) [text width=13.5cm, minimum width=14cm]
        at (0.5,-8.5) [anchor=north west]
        {Files used by \texttt{restorecon} and \texttt{setfiles}};

    \draw[arrow] (5,-1.5) -- (5,-2.5);
    \draw[arrow] (4,-3.5) -- (2.25,-7);
    \draw[arrow] (6,-3.5) -- (7.75,-4.5);
    \draw[arrow] (7.75,-5.5) -- node[pos=0.33,right] {\texttt{genhomedircon}}
        (7.75,-7);
\end{tikzpicture}

    \caption{File Context Configuration Files.}
\end{figure}

These files are created when building the policy, see figure
\ref{fig:filecontexts}. All \texttt{.fc} files from the base policy and from
policy modules are used to build the \texttt{file\_contexts.template} file. This
file may contain rules having special keywords inside their path, such as
\texttt{HOME\_ROOT}, \texttt{HOME\_DIR}, or \texttt{USER}. All rules without
special keywords are used to build the \texttt{file\_contexts} file which is
used directly by utilities such as restorecon or setfiles.

Rules with special keywords are used to build the \texttt{homedir\_template}
file. These rules are associated with user home directories and need to be
expanded for individual users using the genhomedircon utility. For example the
following \texttt{homedir\_template} entry:
\begin{lstlisting}
HOME_DIR/\.ssh(/.*)?        system_u:object_r:ssh_home_t:s0
\end{lstlisting}
would be expanded to the following rules:
\begin{lstlisting}
/home/[^/]*/\.ssh(/.*)?     system_u:object_r:ssh_home_t:s0
/root/\.ssh(/.*)?           system_u:object_r:ssh_home_t:s0
\end{lstlisting}
Expanded rules are then stored in the \texttt{file\_contexts.homedirs} file and
used by restorecon and setfiles utilities \cite[pp.~134--140]{tsn}.

\subsection{Changing File Context Configuration Files}
The \texttt{file\_contexts.local} file can be changed using the \texttt{semanage
fcontext} command \cite{selinuxguide}. For example:
\begin{lstlisting}
# semanage fcontext -a -t samba_share_t /etc/myfile
# semanage fcontext -l -C
SELinux fcontext    type        Context
/etc/myfile         all files   system_u:object_r:samba_share_t:s0
\end{lstlisting}
In this example, a~new file context entry was added. The rule states that the
\texttt{/etc/myfile} file should obtain the
\texttt{system\_u:object\_r:samba\_share\_t:s0} context.

% ------------------------------------------------------------------------------
% TOPIC: Auditing
% ----------------------------------------
% STATUS: 4
% 2.16 AUDITING SELINUX EVENTS                                          Y
%     2.16.1 AVC Audit Events                                           Y
%     2.16.2 General SELinux Audit Events                               Y
\section{Auditing Security Events}

The \emph{Linux Audit System} provides an auditing system for tracking
security-relevant system events. It is used to track file access, monitor system
calls, record commands run by the user, record failed login attempts and others
\cite{secguide}. The Linux Audit System does not provide additional security by
itself, it can be only used to discover security violations.

The Linux Audit System consists of kernel and user space parts. The kernel
filters the events and sends them to the \emph{audit daemon}. The audit daemon
writes the received events to a~log file then. There are several user space
tools used for interacting with the audit system and for working with the log
file.

\subsection{Auditing SELinux Denials}

In Fedora and RHEL, SELinux uses the Linux Audit System to log security events.
When a~process tries to perform an operation without the permissions, an
\emph{Access Vector Cache} (AVC) denial message is logged using the audit daemon
\cite{selinuxguide}. This message can be processed by tools such as
setroubleshoot or audit2allow then.

Each AVC denial message contains information about the source context (the
context of the process), the object class (for example a~file), and the target
context (the context of the object). For example, when an \texttt{httpd} process
running with the \texttt{unconfined\_u:system\_r:\allowbreak httpd\_t:s0}
context is trying to perform the \texttt{getattr} operation on the
\texttt{/var/www/html/\allowbreak file1} file with the
\texttt{system\_u:object\_r:samba\_share\_t:s0} context and fails, the following
AVC denial message is generated:

\begin{lstlisting}
type=AVC msg=audit(1223024155.684:49): avc:  denied  { getattr }
for pid=2000 comm="httpd" path="/var/www/html/file1" dev=dm-0
ino=399185 scontext=unconfined_u:system_r:httpd_t:s0
tcontext=system_u:object_r:samba_share_t:s0 tclass=file
\end{lstlisting}

% ------------------------------------------------------------------------------
% TOPIC: Troubleshooting
% ----------------------------------------
% STATUS: 4
% 11. Troubleshooting
%     11.1. What Happens when Access is Denied
%     11.2. Top Three Causes of Problems
%         11.2.1. Labeling Problems
%         11.2.2. How are Confined Services Running?
%         11.2.3. Evolving Rules and Broken Applications
%     11.3. Fixing Problems
%         11.3.1. Linux Permissions
%         11.3.2. Possible Causes of Silent Denials
%         11.3.3. Manual Pages for Services
%         11.3.4. Permissive Domains
%         11.3.5. Searching For and Viewing Denials
%         11.3.6. Raw Audit Messages
%         11.3.7. sealert Messages
%         11.3.8. Allowing Access: audit2allow
\section{Troubleshooting SELinux}
% SELinux denial - what happens, logs
% auditd
% setroubleshootd
% labeling problems
% services running with wrong context
% ports
% linux perms
% silent denials - dontaudit rules
% permissive domains
% viewing denials
% sealert
% audit2allow

When SELinux denies an access that is requested by a~process, the process may
fail to function normally and reports error or crashes. Determining if the
failure is related to SELinux is done by switching whole SELinux or just one
domain into the permissive mode. For example, for debugging \texttt{httpd}, it
is advised to set the \texttt{httpd\_t} domain into the permissive mode:
\begin{lstlisting}
# semanage permissive -a httpd_t
\end{lstlisting}
SELinux denials caused by the \texttt{httpd\_t} domain would still be logged but
not enforced.

On Fedora and RHEL, SELinux denials are analyzed by the \texttt{setroubleshootd}
daemon that provides suggestions for resolving the problem using various
plugins. Majority of problems are caused by missing \texttt{allow} rules in
policy. To add missing rules, the audit2allow utility is used. However, not
every problem requires new policy rules, some issues may require relabeling of
objects using restorecon or semanage.

% ------------------------------------------------------------------------------
\section{The audit2allow Utility}
% STATUS: 4
\label{audit2allow}
The audit2allow is a~user space tool that scans the AVC messages and generates
SELinux policy snippets based on them.

\subsection{Purpose of audit2allow}
% STATUS: 4
The audit2allow utility is designed for both system administrators and SELinux
policy developers. System administrators use audit2allow to analyze SELinux
denials and to create new policy modules. When suitable, the audit2allow utility
suggests other options to resolve denials, such as turning on a~boolean (see
section \ref{booleans}).

Policy developers use audit2allow to create basis for new policy modules for
their products. When writing a~policy for their program, they can run the
programs test suite in the permissive mode, collect AVC denials, create a~policy
module, and then manually edit the policy module. Policy developers can use
the \texttt{-{}-reference} option to generate the policy using predefined
macros.

\subsection{Basic Mode of Operation}
% STATUS: 4
In default mode, audit2allow scans AVC denial messages and generates policy
rules which allow operations that were denied. For example, when the
\texttt{httpd} process tries to perform the \texttt{getattr} operation on the
\texttt{/var/www/html/file1} file, the following AVC message is generated:
\begin{lstlisting}[escapechar=\%]
type=%\textbf{AVC}% msg=audit(1223024155.684:49): avc:  denied  { %\textbf{getattr}% }
for pid=2000 comm="%\textbf{httpd}%" path="%\textbf{/var/www/html/file1}%" dev=dm-0
ino=399185 scontext=%\textbf{unconfined\_u:system\_r:httpd\_t:s0}%
tcontext=%\textbf{system\_u:object\_r:samba\_share\_t:s0}% tclass=%\textbf{file}%
\end{lstlisting}
The audit2allow utility would generate the following policy rule:
\begin{lstlisting}
allow httpd_t samba_share_t:file getattr;
\end{lstlisting}
The audit2allow utility is able to process multiple AVC denial messages, deal
with duplicates, and output all rules based on the fields in AVC denial
messages.

\subsection{Command-Line Options}
% STATUS: 4
The audit2allow utility is able to read AVC messages from stdin, dmesg, audit
log, or an arbitrary file (see \texttt{-{}-dmesg}, \texttt{-{}-all}, and
\texttt{-{}-input} options). There is \texttt{-{}-boot} option which loads only
the messages generated since last boot and \texttt{-{}-lastreload} option which
loads only the messages since last SELinux policy reload.

The audit2allow utility can output the policy rules directly to stdout or
a~file, or create a~policy module which can be loaded directly into the policy
(see \texttt{-{}-module}, \texttt{-M}, and \texttt{-{}-output} options).

The audit2allow utility is using the currently loaded policy (or another policy
specified with the \texttt{-{}-policy} option) to get more information about the
denials. For example, audit2allow suggests turning on a~boolean that would allow
the denied operations.

When run with the \texttt{-{}-reference} option, audit2allow tries to match the
denials against defined interfaces. An example of audit2allow output without the
\texttt{-{}-reference} option:
\begin{lstlisting}
#============= httpd_t ==============
allow httpd_t samba_share_t:file getattr;
\end{lstlisting}
An example of audit2allow output with the \texttt{-{}-reference} option:
\begin{lstlisting}
require {
	type httpd_t;
}

#============= httpd_t ==============
samba_read_share_files(httpd_t)
\end{lstlisting}
The audit2allow found an interface which contained the same allow rule.
Interfaces create more readable code but can contain more rules that are
necessary.

The \texttt{-{}-why} option does not output any policy rules but provides a~text
description of why the access was denied. An example of \texttt{audit2allow
-{}-why} output:
\begin{lstlisting}
type=AVC msg=audit(1223024155.684:49): avc: denied { getattr }
for pid=2000 comm="httpd" path="/var/www/html/file1" dev=dm-0
ino=399185 scontext=unconfined_u:system_r:httpd_t:s0
tcontext=system_u:object_r:samba_share_t:s0 tclass=file

    Was caused by:
        Missing type enforcement (TE) allow rule.

        You can use audit2allow to generate a~loadable module
        to allow this access.
\end{lstlisting}

The \texttt{-{}-dontaudit} option generates \texttt{dontaudit} rules instead of
\texttt{allow} rules (see section~\ref{avrules}).

\subsection{How Does audit2allow Work}
% STATUS: 4
The audit2allow collects audit messages from various sources first. The messages
are stored based on their type and then parsed. Each AVC denial message is
analyzed together with a~binary policy file to find out the reason of denial.

From AVC denial messages, source contexts, target contexts, object classes, and
permissions are extracted and converted into \emph{access vector sets}. Each
access vector in the set contains a~unique combination of a~source context,
a~target context, and object class. Permissions from multiple AVC messages are
merged into one access vector set. An example of an access vector set:
\begin{lstlisting}[language=Python]
{
  {
    src: 'unconfined_u:system_r:httpd_t:s0',
    tgt: 'system_u:object_r:samba_share_t:s0',
    cls: 'file',
    perms: [ 'getattr', 'open' ]
  },
  {
    src: 'unconfined_u:system_r:httpd_t:s0',
    tgt: 'system_u:object_r:sssd_conf_t:s0',
    cls: 'file',
    perms: [ 'getattr' ]
  }
}
\end{lstlisting}

Each access vector is converted into an allow rule then. After that all rules
are printed to the output. An example:
\begin{lstlisting}
allow httpd_t samba_share_t:file { getattr open };
allow httpd_t sssd_conf_t:file getattr;
\end{lstlisting}

The \texttt{module} and \texttt{require} statements (see section \ref{modules})
may be optionally written to the output. Various other information is stored
during the processing. The audit2allow prints comments with helpful messages.

\subsection{Implementation of audit2allow}
% STATUS: 4
\label{implementation}
The audit2allow utility is a~part of the SELinux user space. It is written mostly
in Python, with several parts written in C. It uses sepolgen and sepolicy Python
packages and libselinux and libsepol libraries.

The main script, \texttt{audit2allow}, parses command-line options, retrieves
audit messages, and prints the output. The main logic of converting AVC denial
messages to access vector rules is implemented in the sepolgen package.

The sepolgen package contains the following modules:
\begin{description}
    \item [\texttt{audit.py}] defines classes for various audit messages,
        contains audit message parser.
    \item [\texttt{access.py}] defines access vectors and access vector sets.
    \item [\texttt{policygen.py}] creates policy rules based on access vectors.
    \item [\texttt{refpolicy.py}] contains classes that represent the policy
        statements.
    \item [\texttt{output.py}] outputs the generated rules.
\end{description}
There are several other modules which are either not significant (e.g. the
\texttt{utils.py} module) or used only to generate policy using the interfaces
(e.g. the \texttt{interfaces.py} package).

\subsubsection{The audit2allow Script}
% STATUS: 4
% parse options
% load policy
    % init the audit2why C module
% read input
    % creates audit parser
    % reads messages
    % feeds them to the parser
% process input
    % filters messages if neccessary
    % converts messages to access vectors
% output
    % output audit2why if selected
    % creates policy generator
    % sets options to the generator
    % adds AVs to the generator
    % writes the output

The main script does the following steps:
\begin{enumerate}
    \item Parses command-line arguments and checks potential conflicts.
    \item Reads audit messages. Creates an \texttt{AuditParser} instance and
        feeds it with the messages.
    \item Filters the messages (if specified by the \texttt{-{}-type} option)
        and converts them to access vectors.
    \item Creates a~\texttt{PolicyGenerator} instance, feeds it with the access
        vectors, and converts them to policy rules.
    \item Writes the output.
\end{enumerate}

\subsubsection{The audit.py Module}
% STATUS: 4
% convenience functions
% AuditMessage class
    % base class for all messages
% InvalidMessage class
% PathMessage class
% PolicyLoadMessage class
% DaemonStartMessage class
% ComputeSidMessage class
% AVCMessage class
    % from_split_string
    % analyze
% AuditParser class
    % parse_file, parse_string
    % to_role, to_access
% AVCTypeFilter
% ComputeSidTypeFilter

The \texttt{audit.py} module is used for parsing audit messages. It is not
a~general purpose library for parsing audit messages, it is meant to parse
mainly the AVC messages and policy load messages.

The \texttt{AuditParser} class reads strings and creates objects of an
appropriate type for each message. The \texttt{AuditMessage} class is the base
class for all message types. The \texttt{AVCMessage} class represents AVC
denials and is used for generating access vectors.

After parsing of an AVC message, the denial is analyzed in the \texttt{audit2why.c}
module (from the libselinux library). The \texttt{audit2why.c} module tries to
find out the reason of the denial by analyzing the policy. The module is written
in C and uses the libsepol library.

Each message is then converted to an access vector from the \texttt{access}
module. AVC denial messages can be filtered using regular expressions via the
\texttt{AVCTypeFilter} class. Only messages that match the regular expression
are processed.

Policy load messages are important for the \texttt{-{}-lastreload} command-line
option. The \texttt{AuditParser} processes then only messages after the last
policy load message.

\subsubsection{The access.py Module}
% STATUS: 4
% AccessVector
    % basic representation of access
    % single source and target type, single object class, set of permissions
% AccessVectorSet
    % used for storing AVs
    % AVs with same source and target type and class are merged
    % add() adds AV to the set
% RoleTypeSet

The \texttt{access.py} module defines the \texttt{AccessVector} and
\texttt{AccessVectorSet} classes. The access vector is a~basic representation of
an access in SELinux. It contains single source and target type, single object
class, and a~set of permissions. Each AVC denial message can be converted into
an access vector. For example this AVC denial message:
\begin{lstlisting}[escapechar=\%]
type=AVC msg=audit(1223024155.684:49): avc:  denied  { %\textbf{getattr}% }
for pid=2000 comm="httpd" path="/var/www/html/file1" dev=dm-0
ino=399185 scontext=%\textbf{unconfined\_u:system\_r:httpd\_t:s0}%
tcontext=%\textbf{system\_u:object\_r:samba\_share\_t:s0}% tclass=%\textbf{file}%
\end{lstlisting}
would be converted into the following access vector:
\begin{lstlisting}[language=Python]
{
    source_context: 'unconfined_u:system_r:httpd_t:s0',
    target_context: 'system_u:object_r:samba_share_t:s0',
    object_class: 'file',
    permissions: [ 'getattr' ]
}
\end{lstlisting}

Multiple access vectors are aggregated in access vector sets. Access vectors
sharing the same source and target type, as well as the object class are merged
together, so that there are no duplicates. For example, if we add the following
access vector:
\begin{lstlisting}[language=Python]
{
    source_context: 'unconfined_u:system_r:httpd_t:s0',
    target_context: 'system_u:object_r:samba_share_t:s0',
    object_class: 'file',
    permissions: [ 'open', 'read' ]
}
\end{lstlisting}
to the access vector above, they would be merged into the following access
vector (they share the source and target context and the object class):
\begin{lstlisting}[language=Python]
{
    source_context: 'unconfined_u:system_r:httpd_t:s0',
    target_context: 'system_u:object_r:samba_share_t:s0',
    object_class: 'file',
    permissions: [ 'getattr', 'open', 'read' ]
}
\end{lstlisting}

Access vector sets serve as a~basis for generating policy access vector rules
in the \texttt{policygen.py} module.

\subsubsection{The policygen.py Module}
% STATUS: 4
% PolicyGenerator
    % generates policy module from access vectors
    % has several options
    % set_gen_refpol()
    % set_gen_requires()
    % set_gen_explain()
    % set_gen_dontaudit()
    % add_access()
% explain_access()
% call_interface()
% InterfaceGenerator
% gen_requires()
The \texttt{policygen.py} module defines the \texttt{PolicyGenerator} class that
generates a~policy module from access vectors. \texttt{PolicyGenerator} converts
an access vector set into SELinux policy statements. For example, this access
vector:
\begin{lstlisting}[language=Python]
{
    source_context: 'unconfined_u:system_r:httpd_t:s0',
    target_context: 'system_u:object_r:samba_share_t:s0',
    object_class: 'file',
    permissions: [ 'getattr', 'open', 'read' ]
}
\end{lstlisting}
would be converted into the following policy statement:
\begin{lstlisting}
allow httpd_t samba_share_t:file { getattr open read };
\end{lstlisting}

\texttt{PolicyGenerator} uses objects from the \texttt{refpolicy.py} module
to represent policy statements. \texttt{PolicyGenerator} provides several
configuration methods:
\begin{description}
    \item [\texttt{set\_gen\_refpol()}] turns on generating of interfaces.
    \item [\texttt{set\_gen\_requires()}] turns on generating of the
        \texttt{require} statements that are necessary for creating
        a~standalone policy module (see section \ref{modules}).
    \item [\texttt{set\_gen\_explain()}] turns on adding of comments explaining
        why the policy statements were generated.
    \item [\texttt{set\_gen\_dontaudit()}] turns on generating of the
        \texttt{dontaudit} rules instead of \texttt{allow} rules (see section
        \ref{avrules}).
\end{description}
The output of \texttt{PolicyGenerator} is a~tree-like structure containing
generated policy statements. The \texttt{output.py} module then just prints out
each statement.

\subsubsection{The refpolicy.py Module}
% STATUS: 4
% PolicyBase
% Node
% Leaf
% walktree, walknode, list_to_space_str, list_to_comma_str
% IdSet
% SecurityContext
% ObjectClass
% TypeAttribute, RoleAttribute
% Role, Type, TypeAlias
% Attribute, AttributeRole
% AVRule
% TypeRule, TypeBound
% RoleAllow, RoleType
% ModuleDeclaration
% Conditional, Bool
% InitialSid, GenfsCon, FilesystemUse
% PortCon, NodeCon, NetifCon, PirqCon, IomemCon, IoportCon
% print_tree
% Headers
% Module
% Interface, TunablePolicy, Template, IfDef, InterfaceCall, OptionalPolicy,
% SupportMacros, Require
% ObjPermSet, ClassMap
% Comment
This module contains classes that represent SELinux policy statements. The
\texttt{Node} and \texttt{Leaf} classes are base classes for all policy
statements. Each statement is either a~node that is a~parent of other
statements (for example the \texttt{Module} class), or a~leaf (for example the
\texttt{AVRule} class). The \texttt{refpolicy.py} module contains functions for
traversing trees made out of nodes and leaves. These functions are used when
printing statements in the \texttt{output.py} module.

The \texttt{IdSet} class represents a~set of arbitrary identifiers and is used
by many statements for storing permissions and other sets. The
\texttt{SecurityContext} class represents an SELinux security context. Classes
such as \texttt{TypeAttribute}, \texttt{RoleAttribute}, \texttt{Role},
\texttt{Type}, and others represent policy statements as described in section
\ref{policy} and are used mainly for interface generation.

For the basic operation mode, the following classes are used: \texttt{AVRule},
\texttt{Module}, \texttt{Require}, and \texttt{ModuleDeclaration}. The
\texttt{AVRule} class contains the following attributes:
\begin{description}
    \item [\texttt{src\_types}] is an \texttt{IdSet()} of source types.
    \item [\texttt{tgt\_types}] is an \texttt{IdSet()} of target types.
    \item [\texttt{obj\_classes}] is an \texttt{IdSet()} of object classes.
    \item [\texttt{perms}] is an \texttt{IdSet()} of permissions.
    \item [\texttt{rule\_type}] is one of the following: \texttt{ALLOW},
        \texttt{DONTAUDIT}, \texttt{AUDITALLOW}, or \texttt{NEVERALLOW}.
\end{description}
The \texttt{Module} class serves only as a~node that is parent to all statements
inside a~module. The \texttt{ModuleDeclaration} class represents the
\texttt{module} statement and is generated with the \texttt{-{}-module} option.
The \texttt{Require} class represents the \texttt{require} statement inside
policy modules and is generated with either \texttt{-{}-module} or
\texttt{-{}-require} options.

% ----------------------------------------
% THE SELINUX NOTEBOOK
% 2.11 COMPUTING ACCESS DECISIONS                                       ?
% 2.17 POLYINSTANTIATION SUPPORT                                        N
%     2.17.1 Polyinstantiated Objects                                   N
%     2.17.2 Polyinstantiation support in PAM                           N
%         2.17.2.1 namespace.conf Configuration File                    N
%         2.17.2.2 Example Configurations                               N
%     2.17.3 Polyinstantiation support in X-Windows                     N
%     2.17.4 Polyinstantiation support in the Reference Policy          N
% 2.18 PAM LOGIN PROCESS                                                N
% 2.20 LIBSELINUX LIBRARY                                               N
% 4.12 CONSTRAINT STATEMENTS                                            ?
%     4.12.1 constrain                                                  ?
%     4.12.2 validatetrans                                              ?
%     4.12.3 mlsconstrain                                               ?
%     4.12.4 mlsvalidatetrans                                           ?
% 4.13 MLS STATEMENTS                                                   N
%     4.13.1 sensitivity                                                N
%     4.13.2 dominance                                                  N
%     4.13.3 category                                                   N
%     4.13.4 level                                                      N
%     4.13.5 range_transition                                           N
%         4.13.5.1 MLS range Definition                                 N
%     4.13.6 mlsconstrain                                               N
%     4.13.7 mlsvalidatetrans                                           N
% 4.14 SECURITY ID (SID) STATEMENT                                      N
%     4.14.1 sid                                                        N
%     4.14.2 sid context                                                N
% 4.15 FILE SYSTEM LABELING STATEMENTS                                  N
%     4.15.1 fs_use_xattr                                               N
%     4.15.2 fs_use_task                                                N
%     4.15.3 fs_use_trans                                               N
%     4.15.4 genfscon                                                   N
% 4.18 XEN STATEMENTS                                                   N
%     4.18.1 iomemcon                                                   N
%     4.18.2 ioportcon                                                  N
%     4.18.3 pcidevicecon                                               N
%     4.18.4 pirqcon                                                    N
%     1.5. Additional Resources
% SELINUX GUIDE
% 3. Targeted Policy
%     3.1. Confined Processes
%     3.2. Unconfined Processes
%     3.3. Confined and Unconfined Users
%         3.3.1. The sudo Transition and SELinux Roles
% 4. Working with SELinux
%     4.1. SELinux Packages
%     4.2. Which Log File is Used
%     4.3. Main Configuration File
%     4.4. Permanent Changes in SELinux States and Modes
%         4.4.1. Enabling SELinux
%         4.4.2. Disabling SELinux
%     4.5. Changing SELinux Modes at Boot Time
%     4.11. Information Gathering Tools
%     4.12. Prioritizing and Disabling SELinux Policy Modules
%     4.15. Disabling ptrace()
%     4.16. Thumbnail Protection
% 5. The sepolicy Suite
%     5.1. The sepolicy Python Bindings
%     5.2. Generating SELinux Policy Modules: sepolicy generate
%     5.3. Understanding Domain Transitions: sepolicy transition
%     5.4. Generating Manual Pages: sepolicy manpage
% 6. Confining Users
%     6.1. Linux and SELinux User Mappings
%     6.2. Confining New Linux Users: useradd
%     6.3. Confining Existing Linux Users: semanage login
%     6.4. Changing the Default Mapping
%     6.5. xguest: Kiosk Mode
%     6.6. Booleans for Users Executing Applications
% 7. Securing Programs Using Sandbox
%     7.1. Running an Application Using Sandbox
% 8. sVirt
%     8.1. Security and Virtualization
%     8.2. sVirt Labeling
% 9. Secure Linux Containers
% 10. SELinux systemd Access Control
%     10.1. SELinux Access Permissions for Services
%     10.2. SELinux and journald
% 12. Further Information
%     12.1. Contributors
%     12.2. Other Resources

% ==============================================================================
\chapter{Problems of the audit2allow Utility}
% STATUS: 4
\label{analysis}
Issues with the audit2allow utility can be grouped into the following two areas:
\begin{enumerate}
    \item Several recently introduced types of SELinux policy statements are not
        recognized by the audit2allow utility. Extended permission access vector
        rules that allow more granular control are described in section
        \ref{xpermsanalysis}.
    \item Certain SELinux denials are not the result of missing rules in the
        security policy, they are caused by mislabeled objects. The audit2alow
        utility is designed to solve problems by adding new rules to the policy
        and it is not able to detect mislabeled objects or even change the
        context of the objects. Checking mislabeled files is described in
        section~\ref{mislabeled}, checking mislabeled network objects is
        discussed in section~\ref{networkobjects}.
\end{enumerate}

% ------------------------------------------------------------------------------
\section{Extended Permission Access Vector Rules}
% STATUS: 4
\label{xpermsanalysis}
Since policy version 30, the SELinux policy supports the extended permission
access vector rules (see section \ref{extavrules}). Usage of the extended
permission AV rules introduces situations when audit2allow is not able to detect
the true cause of the denial. As a~result, when using the extended permission AV
rules, audit2allow may suggest rules that do not solve the denial.

\subsection{AVC Denials Caused by Extended Permission AV Rules}
% STATUS: 4
Suppose there are the following rules present in the policy:
\begin{lstlisting}
allow src_t tgt_t : tcp_socket ioctl;
allowxperm src_t tgt_t : tcp_socket ioctl 0x42;
\end{lstlisting}
When a~process tries to call \texttt{ioctl(fd, \textbf{0x1234}, argp)}, the
operation would be denied, because only the \texttt{ioctl(fd, \textbf{0x42},
argp)} system call is allowed. The following AVC denial message would be
generated:
\begin{lstlisting}[escapechar=\%]
type=AVC msg=audit(1515017775.689:1722): avc:  denied  { ioctl } for
pid=14587 comm="test" dev="dm-0" ino=8390105 %\textbf{ioctlcmd=0x1234}%
scontext=unconfined_u:unconfined_r:src_t:s0-s0:c0.c1023
tcontext=unconfined_u:object_r:tgt_t:s0 tclass=tcp_socket permissive=0
\end{lstlisting}
The \texttt{ioctlcmd} field contains the request parameter of the ioctl system
call that was denied. This value can be used to construct an \texttt{allowxperm}
rule to allow this operation.

When used for troubleshooting this AVC denial, audit2allow produces the
following output:
\begin{lstlisting}
#============= src_t ==============

#!!!! This avc is allowed in the current policy
allow src_t tgt_t:tcp_socket ioctl;
\end{lstlisting}
which is not helpful. The users are expected to know about the extended
permissions to assume that the allow rule was overridden.

\subsection{Generating Extended Permission AV Rules in audit2allow}
% STATUS: 4
% example 1: no rule
% example 2: both allow and allow rule
The audit2allow utility does have all the information to generate extended
permission AV~rules. There are two situations that may arise when using the
extended permission AV~rules:
\begin{itemize}
    \item There is neither an \texttt{allow} nor \texttt{allowxperm} rule in the
        policy. The audit2allow utility has two options: either generate only
        the \texttt{allow} rule (current behavior) or generate both
        \texttt{allow} and \texttt{allowxperm} rules. Generating
        \texttt{allowxperm} rules is more secure and provides more granular
        access control. However, for many processes that use lots of different
        ioctl calls it may be inefficient.  Generating the \texttt{allowxperm}
        rules automatically also changes the default output of audit2allow and
        breaks lots of integration tests written using the audit2allow utility.
    \item There are both \texttt{allow} and \texttt{allowxperm} rules in the
        policy. This means that the specific ioctl parameter is not allowed. In
        this case the \texttt{allowxperm} rule should be generated.
\end{itemize}

It is not possible to distinguish these two situations by analyzing the AVC
denial itself, because both denials contain the \texttt{ioctlcmd} field. The
audit2allow utility would need to analyze the binary policy. The audit2allow
utility may rather generate extended permission AV rules in all cases (it is
a~stricter, more secure solution) or only when requested by the user (for
example using a~command-line option, it is a~less secure solution, but it does
not break backward compatibility).

In case of using the command-line option, there is still a~risk, that the users
do not know that they should be using that option. Consider the following
example:
\begin{lstlisting}
#============= src_t ==============

#!!!! This avc is allowed in the current policy
allow src_t tgt_t:tcp_socket ioctl;
\end{lstlisting}
In this example, it is not clear that the denial is caused by an extended
permission AV rule. In situations when the denial would be allowed according to
the binary policy (there is an \texttt{allow} rule) and the AVC denial message
contains the \texttt{ioctlcmd} field, the user should be warned about the
extended permission AV rules. For example:
\begin{lstlisting}
#============= src_t ==============

#!!!! This avc is allowed in the current policy
#!!!! This av rule may have been overridden by extended permission av rule
allow src_t tgt_t:tcp_socket ioctl;
\end{lstlisting}
Support for the extended permissions was implemented as a~part of this thesis,
see section~\ref{xpermsimp}.

% ------------------------------------------------------------------------------
\section{Mislabeled Files}
% STATUS: 4
\label{mislabeled}
SELinux relies on correctly labeled files. Sometimes, if the files get
mislabeled, processes cannot access these files and causes AVC denials. When
used for troubleshooting, audit2allow suggests adding new rules to the policy
instead of changing the file label.

\subsection{AVC Denial Messages Caused by Mislabeled Files}
% STATUS: 4
When a~process is trying to access a~file that is mislabeled, the operation is
usually denied (unless the process has also access to the incorrect label). For
example, when the user moves files from the \texttt{/root} directory to the
\texttt{/var/www/html} directory, the files retain their original label:
\begin{lstlisting}
$ ls /var/www/html
unconfined_u:object_r:httpd_sys_content_t:s0 index.html
       unconfined_u:object_r:admin_home_t:s0 my_file.html
\end{lstlisting}
The \texttt{index.html} file has the correct label, but the
\texttt{my\_file.html} file has an incorrect one. The \texttt{httpd} process
cannot access files labeled \texttt{admin\_home\_t}, because there are no allow
rules in the policy for this operation:
\begin{lstlisting}
$ sesearch -A -s httpd_t -t admin_home_t -c file -p read
(nothing)
\end{lstlisting}

As a~result, when trying to view \texttt{my\_file.html}, a~similar AVC denial
message is generated:
\begin{lstlisting}
type=AVC msg=audit(1226270358.848:238): avc:  denied  { read }
for  pid=13349 comm="httpd" ino=8390105 name="my_file.html"
dev=dm-0 ino=218171 scontext=system_u:system_r:httpd_t:s0
tcontext=system_u:object_r:admin_home_t:s0 tclass=file
\end{lstlisting}
In this case, there is the \texttt{ino} field, which contains the inode number
associated with the denial, and the \texttt{name} field, which contains the name
of the file (but not the full path). In cases of the \texttt{getattr} denial,
the \texttt{path} field is present.

\subsection{Solving Problems With Mislabeled Files}
The restorecon utility uses the \texttt{file\_contexts} files to get the default
security contexts of files (see section \ref{filecontexts}). For example, when
the \texttt{/var/www/html/my\_file.html} file is mislabeled, the denial should
be fixed by running restorecon on the file:
\begin{lstlisting}
# restorecon -v /var/www/html/my_file.html
Relabeled /var/www/html/my_file.html from unconfined_u:object_r:
admin_home_t:s0 to unconfined_u:object_r:httpd_sys_content_t:s0
\end{lstlisting}

When using audit2allow, the following rules are generated:
\begin{lstlisting}
allow httpd_t admin_home_t:file getattr;
\end{lstlisting}
This means that the \texttt{httpd} process would gain access to all root's
files. This solution is not secure because it contains unnecessary rules to be
added to the policy and it does not solve the real problem.

\subsection{Improving audit2allow}
% STATUS: 4
The audit2allow utility should detect when the AVC message is caused by
a~mislabeled file and suggest a~solution using the restorecon utility.

There are three fields in the AVC message that can be used to detect if the file
was mislabeled: \texttt{path}, \texttt{name} and \texttt{inode}. When the
\texttt{path} field is present, audit2allow can run \texttt{matchpathcon} to get
the default context of the file and compare it with the actual file context.

In many cases, the \texttt{path} field is not present, only an inode number and
the name of the file (without the full path). In this case, it is difficult to
find the full path of the file. Ryan Hallisey created a~solution
\cite{restoreconpullreq} that uses the \texttt{locate} utility to get all files
matching the name and then it stats these files to get the inode number. This
solution is only partial, it does not work on files that are not indexed in the
database created by \texttt{updatedb}.

% ------------------------------------------------------------------------------
\section{Labeling Network Ports, Nodes, and Interfaces}
\label{networkobjects}
The SELinux policy supports labeling of TCP and UDP ports, network nodes
(represented by IP addresses and subnet masks), and network interfaces (e.g.
\texttt{eth0}).

\subsection{Network Ports}
% STATUS: 4
SELinux can enforce binding to system ports. For example, in Fedora 27, there
are several hundred \texttt{portcon} rules that label TCP and UDP ports.
An example:
\begin{lstlisting}
$ seinfo --portcon

Portcon: 615
   portcon tcp 1-511 system_u:object_r:reserved_port_t:s0
   portcon tcp 7 system_u:object_r:echo_port_t:s0
   portcon tcp 21 system_u:object_r:ftp_port_t:s0
   portcon tcp 22 system_u:object_r:ssh_port_t:s0
   portcon tcp 53 system_u:object_r:dns_port_t:s0
   portcon tcp 80 system_u:object_r:http_port_t:s0
   portcon udp 1-511 system_u:object_r:reserved_port_t:s0
   portcon udp 1 system_u:object_r:inetd_child_port_t:s0
   portcon udp 7 system_u:object_r:echo_port_t:s0
   portcon udp 53 system_u:object_r:dns_port_t:s0
   portcon udp 67 system_u:object_r:dhcpd_port_t:s0
   ...
\end{lstlisting}
Portcon rules can overlap, for example a~TCP port number 80 is labeled
\texttt{http\_port\_t} but also \texttt{reserved\_port\_t} because it is in the
range 1--511. Each port has either a~domain-specific label or one of the
following labels (based on a~range):

\begin{tabular}{l l}
    1--511 & \texttt{reserved\_port\_t} \\
    512--1023 & \texttt{hi\_reserved\_port\_t} \\
    1024--32767 & \texttt{unreserved\_port\_t} \\
    32768--61000 & \texttt{ephemeral\_port\_t} \\
    61001--65535 & \texttt{unreserved\_port\_t} \\
\end{tabular}

When a~process tries to bind to a~port and it is denied by SELinux, an AVC
denial message is generated. For example:
\begin{lstlisting}[escapechar=\%]
type=AVC msg=audit(1516026512.648:4191): avc:  denied  { name_bind } for
pid=6116 comm="test" %\textbf{src=43}% scontext=unconfined_u:unconfined_r:my_app_t:s0
tcontext=system_u:object_r:reserved_port_t:s0
tclass=tcp_socket permissive=0
\end{lstlisting}

The proper way how to allow the process to bind on the port number 43 would be
to label this port with an~application-specific context using either
a~\texttt{portcon} rule or the \texttt{semanage port} command. For example:
\begin{lstlisting}
# semanage port --add -t system_u:object_r:my_app_port_t:s0 -p TCP 43
\end{lstlisting}

When used for troubleshooting this denial, the audit2allow utility suggests
adding the following rule to the policy:
\begin{lstlisting}[language=te]
allow my_app_t reserved_port_t:tcp_socket name_bind;
\end{lstlisting}
This rule would grant \texttt{my\_app\_t} access to all reserved ports which is
unnecessary and potentially insecure.

The ports can be labeled using the \texttt{portcon} rules, but as of policy
version 31, these rules are not valid in a~policy module, only in the base
policy. So audit2allow would not be able to generate the \texttt{portcon} rules
directly.  Another way of labeling ports is via the \texttt{semanage port}
command. The audit2allow should suggest using the \texttt{semanage port} command
when appropriate.

\subsection{Network Nodes}
% STATUS: 4
SELinux is capable of labeling network nodes. For example, there can be rules
that allow processes to communicate only on a~private LAN or even only on the
local host. Attempts to violate these rules would then produce AVC denial
messages that contain IP addresses of the nodes.

The proper solution would be to modify a~label of a~certain subnet of the
network, for example using the \texttt{semanage node} command. The AVC denial
messages provide only IP addresses. As IP addresses can often change, labeling
a~single network node would not be useful. Labeling of network nodes is not used
on Fedora and RHEL. Improving the audit2allow utility to suggest
\texttt{semanage node} commands would not make a~big impact on the security.

\subsection{Network Interfaces}
% STATUS: 4
SELinux is capable of labeling network interfaces (such as \texttt{eth0}). In
Fedora and RHEL, labeling of network interfaces is not used. Improving
audit2allow to suggest \texttt{semanage interface} commands would not make a~big
impact on security.

% ==============================================================================
\chapter{Implementing Extensions of the audit2allow Utility}
% STATUS: 4
\label{impl}
From the list of possible improvements to audit2allow, the following
improvements have been implemented:
\begin{itemize}
    \item Support for extended permissions. The audit2allow utility can now
        detect denials that may be caused by extended permission AV rules. With
        the \texttt{-{}-xperms} option, audit2allow generates extended AV rules.
        Implementation details of this extension are described in section
        \ref{xpermsimp}.
    \item Checking mislabeled files. The audit2allow utility now parses the
        \texttt{path} field in AVC denial messages and checks if files have the
        default context. When the context in the AVC denial message is different
        than the default one, audit2allow produces a warning. This extension is
        described in section \ref{mislabeledimp}.
\end{itemize}

% ------------------------------------------------------------------------------
\section{Extended Permissions}
% STATUS: 4
\label{xpermsimp}
Main script \texttt{audit2allow} and the \texttt{audit.py}, \texttt{access.py},
\texttt{policygen.py}, \texttt{refpolicy.py} modules were modified to support
extended permissions. Support for extended permission AV rules with the
\texttt{-{}-reference} option was not implemented.

In main script, a~new command-line option \texttt{-{}-xperms} was added to turn
on generating of the extended permission AV rules. This option turns on extended
permission AV rules generating in \texttt{PoligyGenerator}.

\subsection{Parsing AVC Denial Messages}
% STATUS: 4
Despite extended permissions being a~general concept, AVC denial messages
generated by the Linux Audit System contain operation-specific field
\texttt{ioctlcmd}. In the future, with introduction of new operation to the
extended permissions, the \texttt{audit.py} module will need to be updated.

The \texttt{audit.py} module was extended to parse the \texttt{ioctlcmd} field
in AVC denial messages. The \texttt{ioctlcmd} field is then converted to fit the
general concept of extended permissions and passed to the access vector set. The
following code in the \texttt{AuditParser.to\_access()} method converts
operation-specific fields to generic extended permission dictionary:
\pagebreak
\begin{lstlisting}[language=Python]
if avc.ioctlcmd:
    xperm_set = refpolicy.XpermSet()
    xperm_set.add(avc.ioctlcmd)
    xperms = { "ioctl": xperm_set }
\end{lstlisting}

\subsection{Storing Extended Permissions}
% STATUS: 4
AVC denial messages are converted to access vectors (see section
\ref{implementation}). Two messages that contains the same source and target
context and the object class are merged together. Denials caused by the ioctl
system call produce AVC denial messages containing the \texttt{ioctlcmd} field.
These messages can be treated the same way as any other AVC denial message, but
for generating extended permission AV rules, the \texttt{ioctlcmd} field must be
stored. The \texttt{AccessVector} class was extended to store the
\texttt{ioctlcmd} field and potentially any other extended permission fields
introduced in the future.

Single access vector may be product of several AVC denial messages that contain
different values in the \texttt{ioctlcmd} field. For the purpose of representing
extended permission values, new class \texttt{XpermSet} was implemented in the
\texttt{refpolicy.py} module. The \texttt{XpermSet} class supports merging of
multiple values. It is likely that in the future new extended permission
operation will be introduced. Access vector must store values for different
operations. For this purpose, \texttt{XpermSet} objects are stored in
a~dictionary where an operation is the key, for example:
\begin{lstlisting}
{
    'ioctl': <refpolicy.XpermSet() object>,
    'other_command': <refpolicy.XpermSet() object>,
    'another_command': <refpolicy.XpermSet() object>,
}
\end{lstlisting}

When merging two access vectors, the permission sets are merged using the
\texttt{union()} method. Extended permission dictionary is merged using the
following code from the \texttt{AccessVector.merge()} method:
\begin{lstlisting}[language=Python]
for op in av.xperms:
    if op not in self.xperms:
        self.xperms[op] = refpolicy.XpermSet()
    self.xperms[op].extend(av.xperms[op])
\end{lstlisting}
The \texttt{av} variable contains access vector that is to be merged with
the \texttt{self} access vector.

\subsection{Representation of Extended Permission AV Rules}
% STATUS: 4
Extended permission access vector rules are represented by new
\texttt{AVExtRule} class from the \texttt{refpolicy.py} module. This class is
very similar to the \texttt{AVRule} class, but contains attributes specific to
extended permission AV rules. The \texttt{operation} attribute is a~string
identifying the operation, e.g. 'ioctl'. The \texttt{xperms} attribute is an
\texttt{XpermSet} instance. Method \texttt{to\_string()} prints out the rule.
An example of an extended permission AV rule:
\begin{lstlisting}
allowxperm my_app_t my_socket_t : tcp_socket ioctl { 20 30 0x40 50-60 };
\end{lstlisting}

The \texttt{AVExtRule} class is initialized from an access vector using the
\texttt{from\_av()} method. This method contains the \texttt{op} parameter to
select which extended permission operation from the access vector should appear
in the rule.

Without extended permissions, each access vector can be converted into a~single
AV rule. With extended permissions attached to the access vector, to fully
convert an access vector to the policy rules, there need to be one AV rule and
possibly several extended permission AV rules. For example, this access vector:

\begin{lstlisting}[language=Python]
{
    source_context: 'unconfined_u:system_r:src_t:s0',
    target_context: 'system_u:object_r:tgt_t:s0',
    object_class: 'tcp_socket',
    permissions: { 'getattr', 'ioctl' }
    extended_permissions: {
        'ioctl': { 1, 2, 3 },
        'other_command': { 40, 50, 60 },
        'another_command': { 700, 800, 900 },
    }
}
\end{lstlisting}
would be converted into these policy rules\footnote{Note that as of policy
version 31, only the \texttt{ioctl} operation is supported, operations
\texttt{other\_command} and \texttt{another\_command} were added only as an
example.}:
\begin{lstlisting}
allow src_t tgt_t:tcp_socket { getattr ioctl };
allowxperm src_t tgt_t:tcp_socket ioctl { 1 2 3 };
allowxperm src_t tgt_t:tcp_socket other_command { 40 50 60 };
allowxperm src_t tgt_t:tcp_socket another_command { 700 800 900 };
\end{lstlisting}

\subsection{Generating Extended Permission AV Rules}
% STATUS: 4
A~new configuration method \texttt{set\_gen\_xperms()} was added to the
\texttt{PolicyGenerator} from the \texttt{policygen.py} module, to specify
whether the extended permission AV rules should be generated.

In the old implementation, method \texttt{add\_access()} called the
\texttt{\_\_add\_allow\_rules()} method which generated AV rules for each
access vector set. In new implementation, there are two methods,
\texttt{\_\_add\_av\_rule()} and \texttt{\_\_add\_ext\_av\_rules()}. Both accept
access vector as a~parameter, the \texttt{\_\_add\_av\_rule()} method generates
single standard access vector rule, \texttt{\_\_add\_ext\_av\_rules()} method
generates (possibly) several extended permission access vector rules. Both
methods are called from the \texttt{add\_access()} methods:
\begin{lstlisting}
for av in raw_allow:
    self.__add_av_rule(av)
    if self.xperms and av.xperms:
        self.__add_ext_av_rules(av)
\end{lstlisting}


% ------------------------------------------------------------------------------
\section{Mislabeled Files}
% STATUS: 4
\label{mislabeledimp}
The audit2allow utility was extended to check the default context of a~file if
the \texttt{path} field is present in the AVC denial message. The
\texttt{audit.py} and \texttt{policygen.py} modules were modified.

\subsection{Parsing File Path}
% STATUS: 4
The \texttt{audit.py} module was modified to parse the \texttt{path} field in
AVC denial messages. Only paths found directly in AVC denial messages will be
analyzed later by matchpathcon. As a~result, the context will not be checked in
many cases, because the AVC denials often does not contain full path, only inode
number and file name.

\subsection{Checking Default Context}
% STATUS: 4
In the \texttt{policygen.py} module, a~new option was added to the
\texttt{PolicyGenerator} to turn on or off checking of mislabeled files.
Checking is turned on by default. Each AVC message from each access vector is
checked whether it contains the path. The default context of the path is
obtained via the \texttt{selinux.matchpathcon()} function then. The target
context of the access vector is then compared with the default context. In
a~case of difference, a~comment is added to warn the user about the mislabeled
file. For example:
\begin{lstlisting}
#============= src_t ==============

#!!!! The '/etc/myfile' file has other than the default context
allow src_t tgt_t:file getattr;
\end{lstlisting}

% ==============================================================================
\chapter{Testing Extensions of the audit2allow Utility}
% STATUS: 4
\label{testing}
The functionality of implemented features to audit2allow was tested by extending
existing unit tests and writing integration tests that are focused on
interoperation between audit2allow, SELinux, and Linux Audit System.

% ------------------------------------------------------------------------------
\section{Testing Extended Permissions Implementation}
% STATUS: 4
The unit tests were extended to ensure that the new functionality does not break
the existing code. New test cases were added to test the new features.

\subsection{Testing the audit2allow Script}
% STATUS: 4
In main script, new \texttt{-{}-xperms} and \texttt{-x} option was added to turn
on extended permission AV rules generation.
%\extrarowsep=5pt
\tabulinesep=5pt

\begin{longtabu}{|l|X|} \hline \endfirsthead
    \multicolumn2{|l|}{\texttt{Audit2allowTests} class from the
    \texttt{test\_audit2allow.py} module:}
    \\ \hline
    \texttt{test\_xperms()} & Tests that \texttt{audit2allow -x} produces at
    least one \texttt{allowxperm} rule. \\ \hline
\end{longtabu}

\subsection{Testing the audit.py Module}
% STATUS: 4
In this module, the audit message parser have been modified to recognize new
field in AVC denial messages and to convert the field to extended permissions.

Testing \texttt{AVCMessage.\_\_init\_\_()}:
\begin{longtabu}{|l|X|} \hline \endfirsthead
    \multicolumn2{|l|}{\texttt{TestAVCMessage} class from the
    \texttt{test\_audit.py} module:}
    \\ \hline
    \texttt{test\_defs()} & Tests that \texttt{AVCMessage.ioctlcmd} is
    \texttt{None}.
    \\ \hline
\end{longtabu}

Testing \texttt{AVCMessage.from\_split\_string()}:
\begin{longtabu}{|l|X|} \hline \endfirsthead
    \multicolumn2{|l|}{\texttt{TestAVCMessage} class from the
    \texttt{test\_audit.py} module:}
    \\ \hline
    \texttt{test\_xperms()} & Tests that the \texttt{ioctlcmd} field is parsed.
    \\ \hline
    \texttt{test\_xperms\_invalid()} & Tests a~message with an invalid value in
    the \texttt{ioctlcmd} field
    \\ \hline
    \texttt{test\_xperms\_without()} & Tests a~message without the
    \texttt{ioctlcmd} field.
    \\ \hline
\end{longtabu}

Testing \texttt{AVCMessage.to\_access()}:
\begin{longtabu}{|l|X|} \hline \endfirsthead
    \multicolumn2{|l|}{\texttt{TestAuditParser} class from the
    \texttt{test\_audit.py} module:}
    \\ \hline
    \texttt{test\_parse\_xperms()} & Tests that correct access vectors are
    generated from a~set of AVC denial messages.
    \\ \hline
\end{longtabu}

\subsection{Testing the access.py Module}
% STATUS: 4
In this module, the \texttt{AccessVector} and \texttt{AccessVectorSet} classes
have been extended.

Testing \texttt{AccessVector.\_\_init\_\_()}:
\begin{longtabu}{|l|X|} \hline \endfirsthead
    \multicolumn2{|l|}{\texttt{TestAccessVector} class from the
    \texttt{test\_access.py} module:}
    \\ \hline
    \texttt{test\_init()} & Tests that \texttt{AccessVector.xperms} is
    a~dictionary.
    \\ \hline
\end{longtabu}

Testing \texttt{AccessVector.merge()}: this method must correctly merge
permissions and extended permissions of two access vectors.
\begin{longtabu}{|l|X|} \hline \endfirsthead
    \multicolumn2{|l|}{\texttt{TestAccessVector} class from the
    \texttt{test\_access.py} module:}
    \\ \hline
    \texttt{test\_merge\_noxperm()} & Tests merging two AVs without extended
    permissions.
    \\ \hline
    \texttt{test\_merge\_xperm1()} & Tests merging AV that contains extended
    permissions with AV that does not.
    \\ \hline
    \texttt{test\_merge\_xperm2()} & Tests merging AV that does not contain
    extended permissions with AV that does.
    \\ \hline
    \texttt{test\_merge\_xperm\_diff\_op()} & Tests merging two AVs, both
    containing extended permissions, but with different operations.
    \\ \hline
    \texttt{test\_merge\_xperm\_same\_op()} & Tests merging two AVs, both
    containing extended permissions with the same operation.
    \\ \hline
\end{longtabu}

Testing \texttt{AccessVector.add\_av()}: this method adds an access vector to
the set.
\begin{longtabu}{|l|X|} \hline \endfirsthead
    \multicolumn2{|l|}{\texttt{TestAccessVectorSet} class from the
    \texttt{test\_access.py} module:}
    \\ \hline
    \texttt{test\_add\_av\_first()} & Tests adding the first access vector to
    the access vector set.
    \\ \hline
    \texttt{test\_add\_av\_second()} & Tests adding the second AV to the set
    with the same source and the target context and class.
    \\ \hline
    \texttt{test\_add\_av\_with\_msg()} & Tests adding an audit message.
    \\ \hline
\end{longtabu}

Testing \texttt{AccessVector.add()}: this method just creates an instance of the
\texttt{Access\-Vector} class and passes the AV to the
\texttt{AccessVector.add\_av()} method.
\begin{longtabu}{|l|X|} \hline \endfirsthead
    \multicolumn2{|l|}{\texttt{TestAccessVectorSet} class from the
    \texttt{test\_access.py} module:}
    \\ \hline
    \texttt{test\_add()} & Tests adding access vector to the set.
    \\ \hline
\end{longtabu}

\subsection{Testing the policygen.py Module}
% STATUS: 4
In this module, the \texttt{PolicyGenerator} class was extended to generate
extended permission access vector rules.

Testing \texttt{PolicyGenerator.\_\_init\_\_()}:
\begin{longtabu}{|l|X|} \hline \endfirsthead
    \multicolumn2{|l|}{\texttt{TestPolicyGenerator} class from the
    \texttt{test\_policygen.py} module:}
    \\ \hline
    \texttt{test\_init()} & Tests that extended permission AV rules are not
    generated by default.
    \\ \hline
\end{longtabu}

Testing \texttt{PolicyGenerator.set\_gen\_xperms()}:
\begin{longtabu}{|l|X|} \hline \endfirsthead
    \multicolumn2{|l|}{\texttt{TestPolicyGenerator} class from the
    \texttt{test\_policygen.py} module:}
    \\ \hline
    \texttt{test\_set\_gen\_xperms()} & Tests turning on and off generating of
    extended permission AV rules.
    \\ \hline
\end{longtabu}

Testing \texttt{PolicyGenerator.add\_access()}:
\begin{longtabu}{|l|X|} \hline \endfirsthead
    \multicolumn2{|l|}{\texttt{TestPolicyGenerator} class from the
    \texttt{test\_policygen.py} module:}
    \\ \hline
    \texttt{test\_av\_rules()} & Tests that the new implementation of the
    \texttt{add\_access()} method does not break generating of standard AV
    rules.
    \\ \hline
    \texttt{test\_ext\_av\_rules()} & Tests that correct extended permission AV
    rules are generated from access vectors.
    \\ \hline
\end{longtabu}

\subsection{Testing the refpolicy.py Module}
% STATUS: 4
The \texttt{XpermSet} and \texttt{AVExtRule} classes were added to represent
extended permission access vector rules.

Testing \texttt{XpermSet} class: this class represents extended permission
values. The \texttt{add()} method adds values or ranges of values, the
\texttt{extend()} method combines two \texttt{XpermSet} objects, and the
\texttt{to\_string()} method prints the rule.
\begin{longtabu}{|l|X|} \hline \endfirsthead
    \multicolumn2{|l|}{\texttt{TestXpermSet} class from the
    \texttt{test\_refpolicy.py} module:}
    \\ \hline
    \texttt{test\_init()} & Tests that all attributes are correctly initialized.
    \\ \hline
    \texttt{test\_normalize\_ranges()} & Tests that ranges that are overlapping
    or neighboring are correctly merged into one range.
    \\ \hline
    \texttt{test\_add()} & Tests adding new values or ranges to the set.
    \\ \hline
    \texttt{test\_extend()} & Tests adding ranges from another \texttt{XpermSet}
    object.
    \\ \hline
    \texttt{test\_to\_string()} & Tests printing the values to a~string.
    \\ \hline
\end{longtabu}

Testing \texttt{AVExtRule} class: this class is similar to the \texttt{AVRule}
class. The \texttt{from\_av()} method creates the rule from an access vector and
the \texttt{to\_string()} method prints the rule to a~string.

\begin{longtabu}{|l|X|} \hline \endfirsthead
    \multicolumn2{|l|}{\texttt{TestAVExtRule} class from the
    \texttt{test\_refpolicy.py} module:}
    \\ \hline
    \texttt{test\_init()} & Tests that all attributes are correctly initialized.
    \\ \hline
    \texttt{test\_rule\_type\_str()} & Tests printing the rule type.
    \\ \hline
    \texttt{test\_from\_av()} & Tests creating the rule from an access vector.
    \\ \hline
    \texttt{test\_from\_av\_self()} & Tests creating the rule from an access
    vector that has the same source and target context.
    \\ \hline
    \texttt{test\_to\_string()} & Tests printing the rule.
    \\ \hline
\end{longtabu}

\subsection{Integration Tests of Extended Permissions}
% STATUS: 4

An integration test was written to check audit2allow functionality in real world
situation. First, an SELinux policy module with extended permission AV rules is
loaded. Then the testing program tries to call ioctl on a~file with various
parameters. AVC denials are collected and sent to the audit2allow utility with
different command-line options.

% ------------------------------------------------------------------------------
\section{Testing Implementation of File Contexts Checks}
% STATUS: 4
New unit tests were written to cover the new functionality.

\subsection{Testing the audit.py Module}
% STATUS: 4
In this module, the \texttt{AVCMessage.from\_split\_string()} method was
extended to parse a~path field.
\begin{longtabu}{|l|X|} \hline \endfirsthead
    \multicolumn2{|l|}{\texttt{TestAVCMessage} class from the
    \texttt{test\_audit.py} module:}
    \\ \hline
    \texttt{test\_path()} & Tests that the \texttt{path} field is parsed.
    \\ \hline
\end{longtabu}

\subsection{Testing the policygen Module}
% STATUS: 4
In this module, a~new configuration option was added to the
\texttt{PolicyGenerator}. For the purposes of testing,
\texttt{selinux.matchpathcon()} function used for checking the default context
was replaced. 
\pagebreak
\begin{longtabu}{|l|X|} \hline \endfirsthead
    \multicolumn2{|l|}{\texttt{TestPolicyGenerator} class from the
    \texttt{test\_policygen.py} module:}
    \\ \hline
    \texttt{test\_check\_mislabeled\_nothing()} & Tests no mislabeled files.
    \\ \hline
    \texttt{test\_check\_mislabeled\_one()} & Tests one mislabeled file.
    \\ \hline
\end{longtabu}

\subsection{Integration Tests of File Contexts Checks}
% STATUS: 4
To test checking of mislabeled files, an integration test was written. First,
an SELinux policy module that defines a~new SELinux type is loaded. Test file is
created and labeled with the new type. Testing program then tries to open that
file and fails. AVC denials are collected and sent to the audit2allow utility.
The output is checked for warnings about mislabeled file.

% ==============================================================================
\chapter{Conclusion}
% STATUS: 1
% Závěrečná kapitola obsahuje zhodnocení dosažených výsledků se zvlášť
% vyznačeným vlastním přínosem studenta. Povinně se zde objeví i zhodnocení z
% pohledu dalšího vývoje projektu, student uvede náměty vycházející ze
% zkušeností s řešeným projektem a~uvede rovněž návaznosti na právě dokončené
% projekty (řešené v rámci ostatních bakalářských prací v daném roce nebo na
% projekty řešené na externích pracovištích). 

% TODO: write
% IDEA: result of this thesis is that we know what could be improved and what
% not

% 1. Conclusions: concise statements about your main findings, related to your
%       aims/objectives/hypothesis.
% audit2alow relies on correct labels
% when incorrect label of file or port or whatever, it still tries to suggest
% solutions based on adding new rules to the policy
% audit2allow may provide unfunctional solutions because it does not understands
% new policy statements
The aim of this thesis was to identify situations when the audit2allow utility
provides insecure and too permissive solutions. Several situations were found.
The audit2allow utility relies on objects having the correct labels and it is
not able to detect denials caused by mislabeled objects.

In case of mislabeled files, audit2allow is only partially able to detect
a~mislabeled file. There are ways how to check if the file is mislabeled, but in
most situations, not enough information is logged using the Linux Audit System
to get the full path of the file. Checking of the default file context was
implemented as a~part of this thesis. This improvement warns users that the file
is mislabeled and another tool needs to be used to restore the default context
of the file.

The audit2allow utility is not aware of recently added support for extended
permissions that provide more granular control of permissions given to
processes. When used for troubleshooting SELinux denials caused by extended
permissions, the audit2allow utility is not able to provide a functional
solution.  The support for generating extended permission access vector rules
was implemented as a~part of this thesis. The audit2allow utility can now detect
that the denial was caused by extended permissions and generate extended
permission access vector rules to allow the operations that were previously
denied. SELinux policy developers can now use audit2allow to generate more
restrictive rules as a~basis of a~security policy for their products.

In case of mislabeled TCP or UDP ports, audit2allow can detect that the port
needs an application-specific label. Creating new labels for ports is a~task
that requires certain knowledge of SELinux and the solution is not
straightforward. Suggesting labels for ports can be implemented in the future.

% 2. Contributions to your field of research, stating/restating the significance
%       of what you have discovered. Can include limitations.
% two improvements were implemented
% extended permission support was introduced
% checking of mislabeled files was introduced

% 3. Future research:
%       where to go from here (can include where NOT to go, if your research
%       demonstrated that a~particular approach or avenue was not useful).
% in the future, more improvements can be implemented
% some improvements are not useful
The audit2allow utility can be further improved to detect situations when the
correct solution is to use a~different tool. This thesis serves as a~basis for
discussion about new features that would make troubleshooting problems with
SELinux easier.

  
  % Kompilace po částech (viz výše, nutno odkomentovat)
  % Compilation piecewise (see above, it is necessary to uncomment it)
  %\subfile{projekt-01-uvod-introduction}
  % ...
  %\subfile{chapters/projekt-05-conclusion}


  % Pouzita literatura / Bibliography
  % ----------------------------------------------
\ifslovak
  \makeatletter
  \def\@openbib@code{\addcontentsline{toc}{chapter}{Literatúra}}
  \makeatother
  \bibliographystyle{bib-styles/czechiso}
\else
  \ifczech
    \makeatletter
    \def\@openbib@code{\addcontentsline{toc}{chapter}{Literatura}}
    \makeatother
    \bibliographystyle{bib-styles/czechiso}
  \else 
    \makeatletter
    \def\@openbib@code{\addcontentsline{toc}{chapter}{Bibliography}}
    \makeatother
    \bibliographystyle{bib-styles/englishiso}
  %  \bibliographystyle{alpha}
  \fi
\fi
  \begin{flushleft}
  \bibliography{xzarsk03-Extending-audit2allow-20-literatura-bibliography}
  \end{flushleft}

  % vynechani stranky v oboustrannem rezimu
  % Skip the page in the two-sided mode
  \iftwoside
    \cleardoublepage
  \fi

  % Prilohy / Appendices
  % ---------------------------------------------
  \appendix
\ifczech
  \renewcommand{\appendixpagename}{Přílohy}
  \renewcommand{\appendixtocname}{Přílohy}
  \renewcommand{\appendixname}{Příloha}
\fi
\ifslovak
  \renewcommand{\appendixpagename}{Prílohy}
  \renewcommand{\appendixtocname}{Prílohy}
  \renewcommand{\appendixname}{Príloha}
\fi
%  \appendixpage

% vynechani stranky v oboustrannem rezimu
% Skip the page in the two-sided mode
%\iftwoside
%  \cleardoublepage
%\fi
  
\ifslovak
%  \section*{Zoznam príloh}
%  \addcontentsline{toc}{section}{Zoznam príloh}
\else
  \ifczech
%    \section*{Seznam příloh}
%    \addcontentsline{toc}{section}{Seznam příloh}
  \else
%    \section*{List of Appendices}
%    \addcontentsline{toc}{section}{List of Appendices}
  \fi
\fi
  \startcontents[chapters]
  \setlength{\parskip}{0pt}
  % seznam příloh / list of appendices
  % \printcontents[chapters]{l}{0}{\setcounter{tocdepth}{2}}
  
  \ifODSAZ
    \setlength{\parskip}{0.5\bigskipamount}
  \else
    \setlength{\parskip}{0pt}
  \fi
  
  % vynechani stranky v oboustrannem rezimu
  \iftwoside
    \cleardoublepage
  \fi
  
  % Přílohy / Appendices
  % Tento soubor nahraďte vlastním souborem s přílohami (nadpisy níže jsou pouze pro příklad)
% This file should be replaced with your file with an appendices (headings below
% are examples only)

% Umístění obsahu paměťového média do příloh je vhodné konzultovat s vedoucím
% Placing of table of contents of the memory media here should be consulted with
% a supervisor
\chapter{Contents of Enclosed Memory Media}

\begin{itemize}
    \item selinux repo with extended permissions
    \item selinux repo with checking mislabeled files
    \item package for Fedora 28 with both patches applied
\end{itemize}

\chapter{Manual}

Improvements to audit2allow were implemented as a series of patches to a 2.8-rc1
version of the SELinux userspace upstream repository.

To install patched SELinux userspace, follow the instructions in the upstream
repository (TODO). TODO: installation of packages on Fedora.

TODO: how to run unit tests

TODO: how to integration tests

%\chapter{Konfigurační soubor} % Configuration file

%\chapter{RelaxNG Schéma konfiguračního souboru} % Scheme of RelaxNG configuration file

%\chapter{Plakát} % poster



  
  % Kompilace po částech (viz výše, nutno odkomentovat)
  % Compilation piecewise (see above, it is necessary to uncomment it)
  %\subfile{xzarsk03-Extending-audit2allow-30-prilohy-appendices}
  
\end{document}
