% Tento soubor nahraďte vlastním souborem s přílohami (nadpisy níže jsou pouze pro příklad)
% This file should be replaced with your file with an appendices (headings below
% are examples only)

% Umístění obsahu paměťového média do příloh je vhodné konzultovat s vedoucím
% Placing of table of contents of the memory media here should be consulted with
% a supervisor
\chapter{Contents of the Enclosed CD}

The contents of the enclosed CD:
\begin{description}
    \item [\texttt{/packages}] RPM packages for Fedora 28
    \item [\texttt{/source-packages}] Source RPM packages
    \item [\texttt{/selinux-xperms}] Source codes for extended permission support
    \item [\texttt{/selinux-restorecon}] Source codes for checking mislabeled
        files
    \item [\texttt{/patches}] Source codes as a patches to the upstream
        repository
    \item [\texttt{/tests}] Integration tests
    \item [\texttt{/thesis}] Sources for building the text of the thesis
    \item [\texttt{Extending-audit2allow.pdf}] Text of the thesis
\end{description}

\chapter{Manual}

\section{Upstream Repository}

Improvements to audit2allow were implemented as a series of patches to the
2.8-rc1 version of the SELinux userspace upstream repository. To install the
upstream version of the SELinux userspace, follow the instructions in
\texttt{README}.

The \texttt{/selinux-xperms} directory contains the SELinux userspace repository
with support for extended permissions applied on top of the \texttt{f04d6401}
commit from the upstream repository.

The \texttt{/selinux-restorecon} directory contains the SELinux userspace
repository with support for checking mislabeled files applied on top of the
\texttt{f04d6401} commit from the upstream repository.

The \texttt{/patches} directory contains patches against the \texttt{f04d6401}
commit.

\section{Fedora packages}

Patches for both extended permission support and support for checking mislabeled
files were applied to Fedora userspace packages and can be installed on
Fedora~28. The \texttt{/packages} directory contains RPM packages. The
\texttt{/source-packages} directory contains SRPM packages with source codes.

These packages can be installed directly using the dnf command:
\begin{lstlisting}
# cd packages
# dnf install libselinux-2*.rpm libselinux-utils-2*.rpm
    libsemanage-2*.rpm libsepol-2*.rpm policycoreutils-2*.rpm
    policycoreutils-python-utils-2*.rpm python2-libselinux-2*.rpm
    python3-libselinux-2*.rpm python3-libsemanage-2*.rpm
    python3-policycoreutils-2*.rpm
\end{lstlisting}
or from a Fedora COPR repository:
\begin{lstlisting}
# dnf copr enable janzarsky/selinux-fedora
# dnf install policycoreutils-python-utils
\end{lstlisting}

\section{Tests}
To run unit tests, install the modified packages. Then navigate to the
\texttt{/selinux-xperms} or \texttt{/selinux-restorecon} directory and run tests
from the upstream repository. As a part of this thesis, tests in
\texttt{selinux/\allowbreak python/\allowbreak sepolgen/\allowbreak tests} and 
\texttt{selinux/\allowbreak python/\allowbreak audit2al\-low/\allowbreak
test\_audit2allow} were implemented.

The \texttt{/tests} directory contains integration tests. These tests require
the beakerlib library and the gcc compiler. To run the tests, install modified
packages on the system and execute the \texttt{runtest.sh} files.

