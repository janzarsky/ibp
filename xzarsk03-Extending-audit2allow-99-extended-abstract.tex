% 1 Introduction
% Bakalářská práce rozebírá roli nástroje audit2allow při řešení zamítnutí
% přístupu systémem Security-Enhanced Linux a navrhuje rozšíření nástroje tak,
% aby poskytoval více omezující a~bezpečnější řešení uživateli.
Bakalářská práce se věnuje systému Security-Enhanced Linux, konkrétně nástroji
audit2allow, který analyzuje zprávy o~zamítnutí přístupu a~převádí je na
pravidla bezpečnostní politiky udělující oprávnění, která byla původně
zamítnuta. Cílem práce je analyzovat situace, ve kterých nástroj audit2allow
poskytuje řešení, která udělují zbytečně velké množství oprávnění určitým
procesům.

% 2 Security-Enhanced Linux and audit2allow
%   2.1 Security-Enhanced Linux
%   2.2 Basic Concepts
%   2.3 SELinux Policy
%   2.4 File Contexts
%   2.5 Auditing Security Events
%   2.6 Troubleshooting SELinux
%   2.7 The audit2allow Utility
% TODO: něco o kontextu, uživatelích, rolích, typech, třídách
Kapitola \ref{selinux} představuje SELinux a~nástroj audit2allow. Práce
představuje účel SELinuxu a~přístupy, na kterých je založen, jako je mandatorní
řízení přístupu (mandatory access control, MAC), řízení přístupu založené na
rolích (role-based access control, RBAC) a~vynucení typu (type enforcement, TE).
Základní příkazy bezpečnostní politiky jsou představeny, zejména příkazy
udělující oprávnění, jelikož jsou výstupem nástroje audit2allow. Pozornost je
věnována způsobu, jakým získávají soubory svůj bezpečnostní kontext a jakých
situacích můžou nabýt jiného, než výchozího kontextu. Nástroj audit2allow je
podrobně popsán, jeho účel a~schopnosti, implementační detaily.

% 3 Analysis
%   3.1 Extended Permission Access Vector Rules
%   3.2 Mislabeled Files
%   3.3 Labeling Network Ports, Nodes, and Interfaces
% Druhá část obsahuje analýzu situací, kdy nástroj audit2allow poskytuje řešení,
% která jsou neefektivní a potenciálně nebezpečná.
Kapitola \ref{analysis} analyzuje situace, kdy audit2allow poskytuje řešení,
která jsou neefektivní a~potenciálně nebezpečná. Nástroj audit2allow nerozeznává
tzv. rozšířená oprávnění, která umožňují granulárnějsí způsob přidělování
oprávnění, například dokáží přidělit procesu oprávnění volat jen některá ioctl
systémová volání. Práce rozebírá možnost implementace podpory pro rozšířená
oprávnění. Nástroj audit2allow spoléhá při analýze zamítnutí přístupu na to, že
objekty, ke kterým byl přístup zamítnut, měly správný bezpečnostní kontext.
Práce popisuje, jak by měl nástroj audit2allow detekovat objekty (konkrétně
soubory a~síťové porty, uzly a~rozhraní), které mají špatný bezpečnostní
kontext.

% 4 Implementation
%   4.1 Extended Permissions
%   4.2 Mislabeled Files
% Třetí část popisuje implementaci vybraných rozšíření, která poskytují více
% omezující a~bezpečnější řešení.
Kapitola \ref{implementation} popisuje 

% 5 Functional Testing
%   5.1 Unit Tests of Extended Permissions
%   5.2 Integration Tests of Extended Permissions
%   5.3 Unit Tests of Mislabeled Files
% Poslední část popisuje testování těchto rozšíření.

% 6 Conclusion
