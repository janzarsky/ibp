% 1 Introduction
% Bakalářská práce rozebírá roli nástroje audit2allow při řešení zamítnutí
% přístupu systémem Security-Enhanced Linux a navrhuje rozšíření nástroje tak,
% aby poskytoval více omezující a~bezpečnější řešení uživateli.
Bakalářská práce se zabývá systémem Security-Enhanced Linux, který poskytuje
me\-cha\-nismus mandatorního řízení přístupu v~Linuxu. Práce se věnuje především
nástroji \mbox{audit2allow}, který analyzuje zprávy o~zamítnutí přístupu
a~převádí je na pravidla bez\-peč\-nost\-ní politiky udělující oprávnění, která byla
původně zamítnuta. Cílem práce je analyzovat situace, ve kterých nástroj
audit2allow poskytuje řešení, která udělují zbytečně velké množství oprávnění
určitým procesům. Vybraná rozšíření byla implementována a~v~práci jsou popsána.

% 2 Security-Enhanced Linux and audit2allow
%   2.1 Security-Enhanced Linux
%   2.2 Basic Concepts
%   2.3 SELinux Policy
%   2.4 File Contexts
%   2.5 Auditing Security Events
%   2.6 Troubleshooting SELinux
%   2.7 The audit2allow Utility
% TODO: něco o kontextu, uživatelích, rolích, typech, třídách
Kapitola \ref{selinux} představuje SELinux a~nástroj audit2allow. Práce
představuje účel SE\-Li\-nu\-xu a~přístupy, na kterých je založen, jako je
mandatorní řízení přístupu (mandatory access control, MAC), řízení přístupu
založené na rolích (role-based access control, RBAC) a~vynucení typu (type
enforcement, TE).  Jsou popsány základní příkazy bezpečnostní politiky systému
SELi\-nux, zejména příkazy udělující subjektům oprávnění provádět operace nad
objekty, jelikož tyto příkazy jsou výstupem nástroje audit2allow. Zvýšená
pozornost se věnuje způsobu, jakým získávají soubory svůj bezpečnostní
kontext, protože soubory se špatným bezpečnostním kontextem často způsobují
zamítnutí přístupu, která jsou pak analyzována pomocí nástroje audit2allow. Dále
je představen způsob auditování zpráv o zamítnutí přístupu. Podrobně jsou
popsány účel, schopnosti a implementační detaily nástroje audit2allow.

% 3 Analysis
%   3.1 Extended Permission Access Vector Rules
%   3.2 Mislabeled Files
%   3.3 Labeling Network Ports, Nodes, and Interfaces
% Druhá část obsahuje analýzu situací, kdy nástroj audit2allow poskytuje řešení,
% která jsou neefektivní a potenciálně nebezpečná.
Kapitola \ref{analysis} analyzuje situace, kdy audit2allow poskytuje řešení,
která jsou neefektivní a~potenciálně nebezpečná. Nástroj audit2allow nerozeznává
tzv. rozšířená oprávnění, která umožňují granulárnější způsob přidělování
oprávnění procesům. Rozšířená oprávnění dokáží například přidělit procesu
oprávnění volat jen některá systémová volání ioctl na základě hodnoty parametru
request. Práce rozebírá možnosti implementace podpory pro rozšířená oprávnění
a~ideální výchozí chování nástroje s~ohledem na bezpečnost a~zpětnou
kom\-pa\-tibilitu.
Nástroj audit2allow spoléhá při analýze zamítnutí přístupu na to, že
objekty, ke kterým byl přístup zamítnut, měly správný bezpečnostní kontext.
Práce popisuje problémy, které jsou způsobeny soubory se špatným bezpečnostním
kontextem. Podobně jsou popsány problémy způsobené síťovými porty, uzly
a~rozhraními, které mají špatný bezpečnostní kontext. V práci jsou navrženy
způsoby, jak detekovat objekty se špatným bezpečnostním kontextem.

% 4 Implementation
%   4.1 Extended Permissions
%   4.2 Mislabeled Files
% Třetí část popisuje implementaci vybraných rozšíření, která poskytují více
% omezující a~bezpečnější řešení.
V~rámci práce byla implementována dvě rozšíření nástroje audit2allow, která jsou
po\-psa\-ná v~kapitole \ref{impl}. Pro podporu rozšířených oprávnění bylo potřeba
změnit zpracování vstup\-ních zpráv, vytvořit objekty reprezentující rozšířená
oprávnění a~implementovat ge\-ne\-ro\-vá\-ní pravidel povolující rozšířená
oprávnění. Nástroj audit2allow je nyní schopen rozpoznat zprávy o~zamítnutí
přístupu, které mohou být důsledkem použití rozšířených oprávnění a~vygenerovat
pravidla povolující zamítnutá rozšířená oprávnění. Druhým vylepšením nástroje
audit2allow je kontrola bezpečnostních kontextů souborů. Bylo potřeba detekovat
cesty k~souborům ve vstupních zprávách a~využít knihovnu libselinux pro získání
výchozích bezpečnostních kontextů. Upozornění na soubor se špatným bezpečnostním
kontextem bylo přidáno jako komentář do výstupu audit2allow. Detekci síťových
portů se špatným kontextem lze implementovat v~budoucnu, implementace detekce
síťových uzlů a~rozhraní se špatným kontextem by neměla velký dopad na zvýšení
bezpečnosti systému.

% 5 Functional Testing
%   5.1 Unit Tests of Extended Permissions
%   5.2 Integration Tests of Extended Permissions
%   5.3 Unit Tests of Mislabeled Files
% Poslední část popisuje testování těchto rozšíření.
Testování implementovaných rozšíření je popsáno v~kapitole~\ref{testing}. Změny
v~kódu a~nové třídy jsou pokryty jednotkovými testy, které jsou stručně popsány.
Obě rozšíření jsou testována také pomocí integračních testů, které spouští
audit2allow v~reálných situacích.

% 6 Conclusion
